
\documentclass{beamer}

\usetheme[secheader]{Boadilla}
\setbeamertemplate{footline} {
  %\leavevmode%
  \hbox{%
  \begin{beamercolorbox}[wd=.5\paperwidth,ht=2.25ex,dp=1ex,left]{author in head/foot}%
    \usebeamerfont{author in head/foot}\hspace*{2ex}\insertshortauthor~~(adraeger@cern.ch)
  \end{beamercolorbox}%
  \begin{beamercolorbox}[wd=.5\paperwidth,ht=2.25ex,dp=1ex,right]{date in head/foot}%
    \usebeamerfont{date in head/foot}\insertshorttitle,~
    \insertshortdate{}\hspace*{1em}
    \insertframenumber{} / \inserttotalframenumber\hspace*{2ex}
  \end{beamercolorbox}}%
  \vskip0pt%
}
\beamertemplatenavigationsymbolsempty

\usepackage[percent]{overpic}
\usepackage{tikz}
%\usetikzlibrary{positioning,fit,shapes.arrows,shapes.geometric,shapes.misc,shapes.multipart,calc,shadows}
\tikzstyle{every picture}+=[remember picture]
\usepackage{booktabs}
\usepackage{graphicx}
\usepackage{rotating}
\usepackage{wasysym}
\usepackage{marvosym}
\usepackage{subcaption}
\graphicspath{{../../logo/}{figures/}{../../graphic-common/}}

\input{definitions.tex}
\newcommand{\lib}[1]{\tiny #1}

% Title etc
\title[Inc. SUSY searches Meeting]{Status Report from RA2/b \\ Lost-Lepton Background Estimation Methods \\ \& \ttbar, \wpj Background Rejection Studies }
\subtitle{using phys14 Samples}
\author[Arne-Rasmus~Dr\"ager]{
  Arne-Rasmus~Dr\"ager (Uni Hamburg) \\ on behalf of the RA2/b team
}
\date[January 23, 2015]{January 23, 2015
%   \vskip1cm
  \begin{center}
    \includegraphics[height=1.5cm]{Universitaet-Hamburg-Logo.jpg}
    \hskip8cm
    \includegraphics[height=1.5cm]{CMSlogo.jpeg}
  \end{center}
}

% pdflatex packages
\hypersetup{bookmarks=true}
\hypersetup{unicode=false}
\hypersetup{pdftitle={Lost-Lepton RA2/b}}
\hypersetup{pdfauthor={Arne-Rasmus~Dr\"ager}}


\begin{document}
% ==================================================
% --------------------------------------------------
\begin{frame}
  \titlepage
\end{frame}


% --------------------------------------------------
\section{Introduction}


\begin{frame}
  \frametitle{Lost-lepton Motivation}
  \begin{figure}
    \centering
    \begin{subfigure}[b]{0.3\textwidth}
      \includegraphics[width=\textwidth]{figures/jacks_Studies/T1tttt_feyn}
    \end{subfigure}%
    \begin{subfigure}[b]{0.3\textwidth}
      \includegraphics[width=\textwidth]{figures/jacks_Studies/T1bbbb_feyn}
    \end{subfigure}
    \begin{subfigure}[b]{0.3\textwidth}
      \includegraphics[width=\textwidth]{figures/jacks_Studies/T1qqqq}
    \end{subfigure}
    % \label{fig:models}
  \end{figure}
  \begin{itemize}
  \item RA2/b: inclusive hadronic analyses targeting gluino production (T1tttt, T1bbbb, T1qqqq)
  \item In most sensitive bins, main background is \ttbar events with lost-leptons
  \item T1tttt events have leptons $\sim 70 \%$ of the time
  \begin{itemize}
   \item Are rejected if any lepton passes isolation definition
   \item Enter control sample for background estimation methods
  \end{itemize}  
  \end{itemize}
\end{frame}

\begin{frame}
\frametitle{Baseline selection}
\normalsize
\begin{itemize}
 \item $\HT >500 \gev$
 \begin{itemize}
       \item Jets: $\pt>30\gev$, $|\eta|<2.5$
      \end{itemize}
 \item $\MHT >200 \gev$
  \begin{itemize}
       \item Jets: $\pt>30\gev$, $|\eta|<5.0$
      \end{itemize}
 \item $\NJets\ge 4$, \HT jets
 \item $\BTags$= {$0,1,2,\geq3$} CSVM ($>0.814$), $\pt>30\gev$
 \item $\deltaphi_{1,2,3}>0.5,0.5,0.3$
\item Veto Muons: \href{https://twiki.cern.ch/twiki/bin/view/CMSPublic/SWGuideMuonId\#Tight\_Muon}{2012 ``tight'' ID}: $p_T > 10$ GeV, $I_{rel}\; (\Delta R<0.4) < 0.2$    
    \item Veto Electrons: \href{https://twiki.cern.ch/twiki/bin/viewauth/CMS/CutBasedElectronIdentificationRun2\#CSA14\_selection\_conditions\_25ns}{Phys14 POG ID}:  $p_T > 10$ GeV, $I_{rel}\;
      (\Delta R<0.3) < 0.33 / 0.38$
      \item Under study:
      \begin{itemize}


    \item Taus: \href{https://indico.cern.ch/event/359233/contribution/4/material/slides/0.pdf}{Phys14 POG ID}: $p_T > 10$ GeV, $|\eta| < 2.3$,
      chargedIsoPtSum $(\Delta R<0.5)$ < 1.0 GeV (no neutral isolation yet)
    \item Isolated tracks: $p_T > 15$ GeV, $I_{rel}\;(\Delta R<0.3) < 0.1$ -- just charged candidates
      \end{itemize}
\end{itemize}
% \begin{block}{}
% \centering
% \Large
% \end{block}
\end{frame}

\section{Classical Lost-Lepton Method}
\begin{frame}
  \begin{center}
    {\Large
     Classical Lost-Lepton Prediction Method \\(Arne-Rasmus Draeger)}
  \end{center}
\end{frame}
\subsection{Basic concept}
\begin{frame}
  \frametitle{Mainly \ttbar and \wpj events where prompt electrons or muons are lost}
   \begin{figure}
 \centering
  \includegraphics[width = 0.65\textwidth]{figures/lepton_veto_sketch.png}
%  \caption{Awesome figure}
 \end{figure}
      \begin{itemize}
      \item Select a control sample (CS) of exactly one well isolated \\electron, $\mu$ within the acceptance
        \begin{itemize}
        \item Weight each CS event according to efficiencies for each identification step
        \item Predict expected di-leptonic \ttbar contribution to lost-lepton background independently using the same single $e,\mu$ control sample (about 1\%)
        \item Combine independent prediction from single electron and $\mu$ for final prediction of lost-leptons
        \end{itemize}
      \end{itemize}
\end{frame}


\begin{frame}
\begin{itemize}
 \item Signal and other SM processes can contribute to e/$\mu$ control sample
 \item Suppress contamination by requiring trans. mass $\mt < 100 \gev$ \\
\end{itemize}
\vspace{0.5cm}
\hspace{0.5cm}$m_{T} = \sqrt{2 \cdot p_{T}(\mu/e)\cdot \met (1 - \cos(\Delta \Phi))}$

  \begin{columns}
    \begin{column}{0.5\textwidth}

      \begin{itemize}
      \item Removes about 10\% of e/$\mu$ CS due to:
        \begin{itemize}
        \item di-leptonic \ttbar decays
        \item Mismeasured jets
        \item Highly virtual W
        \end{itemize}
      \begin{centering}
      \end{centering}
      \item Correction for as a function of lepton $p_{T}$ \& activity around the lepton (motivation \& definition on the next slides)
      \end{itemize}
      \vspace{0.3cm}
    \end{column}
    \begin{column}{0.5\textwidth}
      \centering
    %  \begin{overpic}[width=0.8\textwidth]{figures/lost-lepton/ControlSample__MTW__MCPrMTWDiLepTTbar+MCPrMTWDiLepW__mu_control_sample.pdf}
%        \begin{overpic}[width=0.95\textwidth]{figures/control-sample/ControlSample_Combined_searchbins__MTW__MCEx_vs_MuPrMTWDiLep+ElecPrMTWDiLep__Baseline.pdf}
\begin{overpic}[width=0.95\textwidth]{figures/control-sample/mtw.png}
       \put(53.93,80){\color{black}\line(0,-1){67}}
    %\put(90,90){\rotatebox{-45}{\scriptsize \Large Arne}}
      \end{overpic}
    \end{column}
  \end{columns}

\end{frame}

%--------------------------------------------------------------------
\subsection{Efficiency Parametrization}
\begin{frame}
 One of the main background arises from \ttbar and \wpj events in which at least one of the involved W decays leptonically thus involving a neutrino invoking the missing transverse momentum and if the involved lepton is not detected entering the search regions. \\
 In order to estimate the amount of events involving a light lepton (electron or muon) which are lost either because the lepton fails the acceptance criteria or being not reconstrcted or failing the isolation defintion two methods are deployed using a single electron or muon control sample seletect from data. \\
 The first method predictes for each of the mentioned criteria separately the amount of lost-leptons using efficiencies obtained from simulated \ttbar and \wpj events by selecting a sample of either a single well isolated muon or electron which obays the transverse mass of less than 100 GeV by correcting first for the amount of di-leptonic ttbar decay contribution to the single lepton control sample which summs up to about 3\% of the total control sample, correcting for the amount of events failing the transverse mass cut of 100 GeV followed by weighting the amount of observed events according to the isolation, reconstruction and out of acceptance efficiencies of the corresponding flavour typ. The same sample is also used to estimate the amount of lost-leptons of the other flavour by taking advantage of lepton universality which is giving to a high precision in W decays. The final step is predicting the amount of di-leptonic \ttbar decays where both leptons are lost. This is only the case in about 1\% of the total amount of expected lost-lepton events for which also the single lepton control sapmle is used acording to a combined over all efficiecny of losing both letpons.\\
 Finally two predictions for out of acceptance, not reconstructed and not isolated muons and electrons is obtained for each flavour control sample independently which is combined to obtain the final prediction of lost-leptons in each of the search bins. \\
 A careful choice of the lepton efficiency variable binning and a compromise between precision and statistical uncertainties of the efficiencies has been studied and optimized for best closure. \\
 Special care has to be taken for the difficulty of high granilarity of the search bins espeically at the high kinemeatic regims eg high \NJets and btags regions resulting in very limited statistics of the control sample. As can be seen in table (ref to talbe with cs sample events for each search bin and expectation) in some bins the control sample statistics decrease to only a few or even one expected event. In order to avoid high fluctuations of direct kinematic properties of the control sample event kinematics the rather classical approach of chosing highly correlated Parametrization of the efficiecny with the corresponding criteria has been avoid to capture events in a more broader spectrum. 
 The chosen activity variable aims at picking up event topologies at a larger scale than just in the very imediate surroundiing ot the lepton to achieve a medium correlation between the likly to lose region of small delta R of the lepton and a close by jet (thus by definition almost always resulting in non-isolated leptons) and the likely to be isaolted large delta R region (relative to the isolation cone defintion of 0.4).  \\
 
 Uncertainties
 reconstruction and isolation: It is not fully decided which reconstruction and isolation efficiecny will be used by either way a tag and probe method will be deployed to either obtain the efficiecny directly from data selection di-leptonic drell yan events or at least validating the simulation power by comparing simulated tag and probe obtained efficiecny to data. 
 acceptance: The uncertainty on the acceptance is mainly driven by the uncertainties on the used PDF sets for the simulated events from which the acceptance efficiecnies have been obtained. In addition JEC contribute at a smaller value as well.
 mt-cut: A conservaitve uncertaintie of about 50 on the mt-cut efficiecny are taking into account accounting for possible poor simulated highly virtual w decays dileptonic contribution and influcence of mis measured jets.
 di-leptonic contribution: The di-leptonic contribution to the lost-lepton background summs up to about 1 \% of the total amound of expected events thus being a small fraction a conservaitve uncertainty of 100 \% has no seizable impact of the resulting combined uncertainties
 
\end{frame}



\setcounter{framenumber}{19}

\end{document}
