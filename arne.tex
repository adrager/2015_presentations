\documentclass{beamer}
%kdfj
\usetheme[secheader]{Boadilla}
\setbeamertemplate{footline} {
  %\leavevmode%
  \hbox{%
  \begin{beamercolorbox}[wd=.5\paperwidth,ht=2.25ex,dp=1ex,left]{author in head/foot}%
    \usebeamerfont{author in head/foot}\hspace*{2ex}\insertshortauthor~~(adraeger@cern.ch)
  \end{beamercolorbox}%
  \begin{beamercolorbox}[wd=.5\paperwidth,ht=2.25ex,dp=1ex,right]{date in head/foot}%
    \usebeamerfont{date in head/foot}\insertshorttitle,~
    \insertshortdate{}\hspace*{1em}
    \insertframenumber{} / \inserttotalframenumber\hspace*{2ex}
  \end{beamercolorbox}}%
  \vskip0pt%
}
\beamertemplatenavigationsymbolsempty

\usepackage[percent]{overpic}
\usepackage{tikz}
%\usetikzlibrary{positioning,fit,shapes.arrows,shapes.geometric,shapes.misc,shapes.multipart,calc,shadows}
\tikzstyle{every picture}+=[remember picture]
\usepackage{booktabs}
\usepackage{graphicx}
\usepackage{rotating}
\usepackage{wasysym}
\usepackage{marvosym}
\usepackage{amssymb}
\usepackage{xcolor}
\graphicspath{{../../logo/}{figures/}{../../graphic-common/}}

\input{definitions.tex}
\newcommand{\lib}[1]{\tiny #1}

% Title etc
\vskip2cm
\title[RA2/b Meeting]{Classical Lost-Lepton Background}
\subtitle{Tag \& Probe Efficiencies (first try) \\ Exclusive Search Bin Definition}
\author[Arne-Rasmus~Dr\"ager]{
  Arne-Rasmus~Dr\"ager(Uni Hamburg)
}
\date[March 17, 2015]{March 17, 2015
  \vskip1cm
  \begin{center}
    \includegraphics[height=1.5cm]{Universitaet-Hamburg-Logo.jpg}
    \hskip8cm
    \includegraphics[height=1.5cm]{CMSlogo.jpeg}
  \end{center}
}

% pdflatex packages
\hypersetup{bookmarks=true}
\hypersetup{unicode=false}
\hypersetup{pdftitle={Lost-Lepton}}
\hypersetup{pdfauthor={Arne-Rasmus~Dr\"ager}}


\begin{document}
% ==================================================
% --------------------------------------------------
\begin{frame}
  \titlepage
\end{frame}
\section{Lost-Lepton Method}
\begin{frame}
 \begin{block}{}
 \centering
 \Large Tag \& Probe Efficiencies (first try)
 \end{block}
\end{frame}
\subsection{Lost-Lepton Method}
\begin{frame}
 \begin{center}
\begin{tikzpicture}
    \node[anchor=south west,inner sep=0] (image) at (0,0) {\includegraphics[width=0.75\textwidth]{figures/Sketches/LostLeptonSketch_mu_pred_full.pdf}};
    \begin{scope}[x={(image.south east)},y={(image.north west)}]
%         \draw[red,ultra thick,rounded corners] (0.62,0.65) rectangle (0.78,0.75);
        \draw[red,ultra thick,rounded corners] (0.60,0.01) rectangle (0.75,0.99); % cordinates unten links(x,y) oben rechts(x,y)
            \draw[blue,ultra thick,rounded corners] (0.40,0.01) rectangle (0.55,0.99); % cordinates unten links(x,y) oben rechts(x,y)
    \end{scope}
\end{tikzpicture}

 \end{center}
 \begin{itemize}
  \item Deriving reco \& iso efficiencies
 \end{itemize}
\end{frame}


\section{Tag \& Probe}
\subsection{Setup}
\begin{frame}
%  \frametitle{Mini Isolation (UCSB approach \href{https://indico.cern.ch/event/368826/contribution/3/material/slides/0.pdf}{Adam talk}) vs Classical Isolation}
\frametitle{Lepton Definition}
\begin{itemize}
 \item Muon:
 \begin{itemize}
  \item Reco/ID: "Tight" ID
  \item Iso: Mini Isolation: Max Cone: 0.2 Min Cone: 0.05 $\delta \beta I(rel)<0.2$
 \end{itemize}
  \item Electron:
 \begin{itemize}
  \item Reco/ID: "Veto" ID
  \item Iso: Mini Isolation: Max Cone: 0.2 Min Cone: 0.05 $\delta \beta I(rel)<0.1$
 \end{itemize}
 \item Tag \& Probe:
 \begin{itemize}
 \item ID:
 \begin{itemize}
  \item Tag: Iso muon(Electron)
  \item Probe: slimmedMuons(slimmedElectrons) $\rightarrow$ Reco/ID muon(electron)
 \end{itemize}
  \item Iso:
 \begin{itemize}
  \item Tag: Iso muon(Electron)
  \item Probe: Reco/ID muon(electron) $\rightarrow$ Iso muon(electron)
 \end{itemize}

 \end{itemize}
\end{itemize}
\end{frame}
\begin{frame}
 \frametitle{Tag \& Probe technicals}
 \begin{itemize}
 \item Apply $\HT>500$ \& $\NJets\geq4$ (similar kinematics to baseline region)
  \item Select (randomly) tag lepton from iso collection
  \item Select all probe leptons except matched to tag
  \item Compute for all invariant mass (Z mass)
  \item Check property (ID/iso criteria)
  \item Use electroweak code for fitting and eff. determination
  \begin{itemize}
   \item Fit function: gaussPlusCubic
  \end{itemize}

 \end{itemize}

\end{frame}
\begin{frame}
 \frametitle{Comparison \ttbar \& \wpj vs Tag \& Probe Efficiencies}
  \begin{columns}

   \begin{column}{0.50\textwidth}
     \begin{itemize}
   \item $\mu$ Iso \ttbar \& \wpj eff.
  \end{itemize}
    \begin{tikzpicture}
    \node[anchor=south west,inner sep=0] (image) at (0,0) {\includegraphics[width=1.\textwidth]{figures/efficiencies/MuIsoPTActivity.pdf}};
    \begin{scope}[x={(image.south east)},y={(image.north west)}]
%         \draw[red,ultra thick,rounded corners] (0.62,0.65) rectangle (0.78,0.75);
%         \draw[red,ultra thick,rounded corners] (0.60,0.01) rectangle (0.75,0.99); % cordinates unten links(x,y) oben rechts(x,y)
    \end{scope}
   \end{tikzpicture}
   \end{column}
   \begin{column}{0.50\textwidth}
   \begin{itemize}
    \item $\mu$ Iso Tag \& Probe eff.
   \end{itemize}

    \begin{tikzpicture}
    \node[anchor=south west,inner sep=0] (image) at (0,0) {\includegraphics[width=1.\textwidth]{figures/efficiencies/MuIsoTagProbe.pdf}};
    \begin{scope}[x={(image.south east)},y={(image.north west)}]
%         \draw[red,ultra thick,rounded corners] (0.62,0.65) rectangle (0.78,0.75);
%         \draw[red,ultra thick,rounded corners] (0.60,0.01) rectangle (0.75,0.99); % cordinates unten links(x,y) oben rechts(x,y)
    \end{scope}
   \end{tikzpicture}
   \end{column}
  \end{columns}
\begin{itemize}
 \item Tag \& Probe efficiencies lower than \ttbar \& \wpj efficiencies.
 \item \ttbar \& \wpj eff. derived from baseline region (all cuts except lepton veto applied)
 \item \ttbar \& \wpj eff. \mindeltaphi applied changes kinematics rejects events with \MHT (neutrino) \& highest \pt jets allied
\end{itemize}

\end{frame}



\begin{frame}
 \frametitle{Resolve difference (idea)}
 \begin{itemize}
  \item \ttbar \& \wpj: Assume main source of \MHT neutrino from W decay
  \item \Zll: Assume tag lepton to be reasonable equivalent to neutrino from W decay
  \begin{itemize}
   \item Calculate \mindeltaphi for tag lepton \pt
  \end{itemize}
 \end{itemize}
\end{frame}
\begin{frame}
 \frametitle{Comparison \ttbar \& \wpj vs Tag \& Probe Efficiencies}
  \begin{columns}

   \begin{column}{0.50\textwidth}
     \begin{itemize}
   \item $\mu$ ID \ttbar \& \wpj eff.
  \end{itemize}
    \begin{tikzpicture}
    \node[anchor=south west,inner sep=0] (image) at (0,0) {\includegraphics[width=1.\textwidth]{figures/efficiencies/MuRecoPTActivity.pdf}};
    \begin{scope}[x={(image.south east)},y={(image.north west)}]
%         \draw[red,ultra thick,rounded corners] (0.62,0.65) rectangle (0.78,0.75);
%         \draw[red,ultra thick,rounded corners] (0.60,0.01) rectangle (0.75,0.99); % cordinates unten links(x,y) oben rechts(x,y)
    \end{scope}
   \end{tikzpicture}
   \end{column}
   \begin{column}{0.50\textwidth}
   \begin{itemize}
    \item $\mu$ ID Tag \& Probe eff.
   \end{itemize}

    \begin{tikzpicture}
     \node[anchor=south west,inner sep=0] (image) at (0,0) {\includegraphics[width=1.\textwidth]{figures/efficiencies/MuRecoTagProbe.pdf}};
    \begin{scope}[x={(image.south east)},y={(image.north west)}]
%         \draw[red,ultra thick,rounded corners] (0.62,0.65) rectangle (0.78,0.75);
%         \draw[red,ultra thick,rounded corners] (0.60,0.01) rectangle (0.75,0.99); % cordinates unten links(x,y) oben rechts(x,y)
    \end{scope}
   \end{tikzpicture}
   \end{column}
  \end{columns}
\begin{itemize}
 \item Tag \& Probe efficiencies higher than \ttbar \& \wpj efficiencies.
 \item \ttbar \& \wpj eff. matching from all in acceptance (gen lepton) including non reco leptons
 \item \Zll eff. start with slimmedMuons/slimmedElectrons (non reco excluded only test for ID)
\end{itemize}

\end{frame}

\section{Exclusive Search Bins \& CS Size}

\begin{frame}
 \begin{block}{}
 \centering
 \Large Exclusive Search Bins \& Single Lep CS Size
 \end{block}
\end{frame}


\begin{frame}
 \begin{itemize}
  \item Constrains on search bin definition (from lost-lepton):
  \begin{itemize}
   \item Sufficient single lepton CS in data desirable, stat. uncertainties same order expected systematic uncertainties
   \item Acceptance efficiencies directly taking from MC, need reasonable statistics for eff map calculation
  \end{itemize}
 \end{itemize}
  \begin{columns}
  \begin{column}{0.50\textwidth}
   \begin{tikzpicture}
    \node[anchor=south west,inner sep=0] (image) at (0,0) {\includegraphics[width=1.\textwidth]{figures/ra2classic/ElecAccSearchBinEff.pdf}};
    \begin{scope}[x={(image.south east)},y={(image.north west)}]
%         \draw[red,ultra thick,rounded corners] (0.62,0.65) rectangle (0.78,0.75);
%         \draw[red,ultra thick,rounded corners] (0.60,0.01) rectangle (0.75,0.99); % cordinates unten links(x,y) oben rechts(x,y)
\put(134,2){\rotatebox{-0}{\tiny $Bin$}}
\put(10,84){\rotatebox{-0}{\small $\epsilon$}}
    \end{scope}
\end{tikzpicture}
  \end{column}
 \begin{column}{0.50\textwidth}
   \begin{tikzpicture}
    \node[anchor=south west,inner sep=0] (image) at (0,0) {\includegraphics[width=1.\textwidth]{figures/miniIsoJack/Closure_Step_By_Step_Elec__Bin__ElecCSMTWDiLepCorrected_vs_ElecCSMTWDiLepCorrectedWPJ+ElecCSMTWDiLepCorrectedTTbar+ElecCSMTWDiLepCorrectedQCD__Purity.pdf}};
    \begin{scope}[x={(image.south east)},y={(image.north west)}]
%         \draw[red,ultra thick,rounded corners] (0.62,0.65) rectangle (0.78,0.75);
%         \draw[red,ultra thick,rounded corners] (0.60,0.01) rectangle (0.75,0.99); % cordinates unten links(x,y) oben rechts(x,y)
    \end{scope}
\end{tikzpicture}
 \end{column}
 \end{columns}
\end{frame}


\begin{frame}
 \frametitle{Conclusion}
 \begin{itemize}
  \item Tag \& Probe:
 \begin{itemize}
  \item Machinery up and running
  \item First try obtaining ID and Iso efficiencies from \Zll seem to be significant lower
  \item Try to achieve better agreement by selecting more similar kinematics
  \end{itemize}
    \item Exclusive Search Bin Definition
  \begin{itemize}
   \item Available MC statistics insufficient for acceptance eff calculation for each search bin
   \item Expected single lepton CS for 4 fb below 1 in some bins (sensitive)
  \end{itemize}

 \end{itemize}

\end{frame}

\begin{frame}
 \begin{block}{}
 \centering
 \Large Backup
 \end{block}
\end{frame}

\begin{frame}
 \frametitle{Comparison \ttbar \& \wpj vs Tag \& Probe Efficiencies}
  \begin{columns}

   \begin{column}{0.50\textwidth}
     \begin{itemize}
   \item e Iso \ttbar \& \wpj eff.
  \end{itemize}
    \begin{tikzpicture}
    \node[anchor=south west,inner sep=0] (image) at (0,0) {\includegraphics[width=1.\textwidth]{figures/efficiencies/ElecIsoPTActivity.pdf}};
    \begin{scope}[x={(image.south east)},y={(image.north west)}]
%         \draw[red,ultra thick,rounded corners] (0.62,0.65) rectangle (0.78,0.75);
%         \draw[red,ultra thick,rounded corners] (0.60,0.01) rectangle (0.75,0.99); % cordinates unten links(x,y) oben rechts(x,y)
    \end{scope}
   \end{tikzpicture}
   \end{column}
   \begin{column}{0.50\textwidth}
   \begin{itemize}
    \item e Iso Tag \& Probe eff.
   \end{itemize}

    \begin{tikzpicture}
     \node[anchor=south west,inner sep=0] (image) at (0,0) {\includegraphics[width=1.\textwidth]{figures/efficiencies/ElecIsoTagProbe.pdf}};
    \begin{scope}[x={(image.south east)},y={(image.north west)}]
%         \draw[red,ultra thick,rounded corners] (0.62,0.65) rectangle (0.78,0.75);
%         \draw[red,ultra thick,rounded corners] (0.60,0.01) rectangle (0.75,0.99); % cordinates unten links(x,y) oben rechts(x,y)
    \end{scope}
   \end{tikzpicture}
   \end{column}
  \end{columns}

\end{frame}

\begin{frame}
 \frametitle{Comparison \ttbar \& \wpj vs Tag \& Probe Efficiencies}
  \begin{columns}

   \begin{column}{0.50\textwidth}
     \begin{itemize}
   \item e ID \ttbar \& \wpj eff.
  \end{itemize}
    \begin{tikzpicture}
    \node[anchor=south west,inner sep=0] (image) at (0,0) {\includegraphics[width=1.\textwidth]{figures/efficiencies/ElecRecoPTActivity.pdf}};
    \begin{scope}[x={(image.south east)},y={(image.north west)}]
%         \draw[red,ultra thick,rounded corners] (0.62,0.65) rectangle (0.78,0.75);
%         \draw[red,ultra thick,rounded corners] (0.60,0.01) rectangle (0.75,0.99); % cordinates unten links(x,y) oben rechts(x,y)
    \end{scope}
   \end{tikzpicture}
   \end{column}
   \begin{column}{0.50\textwidth}
   \begin{itemize}
    \item e ID Tag \& Probe eff.
   \end{itemize}

    \begin{tikzpicture}
     \node[anchor=south west,inner sep=0] (image) at (0,0) {\includegraphics[width=1.\textwidth]{figures/efficiencies/ElecRecoTagProbe.pdf}};
    \begin{scope}[x={(image.south east)},y={(image.north west)}]
%         \draw[red,ultra thick,rounded corners] (0.62,0.65) rectangle (0.78,0.75);
%         \draw[red,ultra thick,rounded corners] (0.60,0.01) rectangle (0.75,0.99); % cordinates unten links(x,y) oben rechts(x,y)
    \end{scope}
   \end{tikzpicture}
   \end{column}
  \end{columns}

\end{frame}


% --------------------------------------------------

\setcounter{framenumber}{14}

\end{document}

