\documentclass{beamer}
%kdfj
\usetheme[secheader]{Boadilla}
\setbeamertemplate{footline} {
  %\leavevmode%
  \hbox{%
  \begin{beamercolorbox}[wd=.5\paperwidth,ht=2.25ex,dp=1ex,left]{author in head/foot}%
    \usebeamerfont{author in head/foot}\hspace*{2ex}\insertshortauthor~~(adraeger@cern.ch)
  \end{beamercolorbox}%
  \begin{beamercolorbox}[wd=.5\paperwidth,ht=2.25ex,dp=1ex,right]{date in head/foot}%
    \usebeamerfont{date in head/foot}\insertshorttitle,~
    \insertshortdate{}\hspace*{1em}
    \insertframenumber{} / \inserttotalframenumber\hspace*{2ex}
  \end{beamercolorbox}}%
  \vskip0pt%
}
\beamertemplatenavigationsymbolsempty

\usepackage[percent]{overpic}
\usepackage{tikz}
%\usetikzlibrary{positioning,fit,shapes.arrows,shapes.geometric,shapes.misc,shapes.multipart,calc,shadows}
\tikzstyle{every picture}+=[remember picture]
\usepackage{booktabs}
\usepackage{graphicx}
\usepackage{rotating}
\usepackage{wasysym}
\usepackage{marvosym}
\usepackage{amssymb}
\usepackage{xcolor}
\usepackage{tabularx}
\usepackage[normalem]{ulem}
\graphicspath{{../../logo/}{figures/}{../../graphic-common/}}

\input{definitions.tex}
\newcommand{\lib}[1]{\tiny #1}

% Title etc
\vskip2cm
\title[RA2/b Meeting]{Update On Validation of Lepton Efficiencies Using a Tag\&Probe Method}
\subtitle{Lepton Isolation\\ Lepton ID/Reconstruction\\ Isolated Tracks}
\author[Arne-Rasmus~Dr\"ager]{
  Arne-Rasmus~Dr\"ager(Uni Hamburg)
}
\date[June 10, 2015]{June 10, 2015
  \vskip1cm
  \begin{center}
    \includegraphics[height=1.5cm]{Universitaet-Hamburg-Logo.jpg}
    \hskip8cm
    \includegraphics[height=1.5cm]{CMSlogo.jpeg}
  \end{center}
}

% pdflatex packages
\hypersetup{bookmarks=true}
\hypersetup{unicode=false}
\hypersetup{pdftitle={Lost-Lepton}}
\hypersetup{pdfauthor={Arne-Rasmus~Dr\"ager}}


\begin{document}
% ==================================================
% --------------------------------------------------
\begin{frame}
  \titlepage
\end{frame}

\section{Classical Lost-Lepton Method}
\begin{frame}
 \begin{block}{}
 \centering
 \Large Tack\&Probe on Isolated $e,\mu,\pi$ Tracks
 \end{block}
\end{frame}
\subsection{Concept}
\begin{frame}
\frametitle{Classical Lost-Lepton Procedure}
 \begin{center}
\begin{tikzpicture}
    \node[anchor=south west,inner sep=0] (image) at (0,0) {\includegraphics[width=0.75\textwidth]{figures/Sketches/LostLeptonSketch_Prediction_IsoTrackReduction.pdf}};
    \begin{scope}[x={(image.south east)},y={(image.north west)}]
%         \draw[red,ultra thick,rounded corners] (0.62,0.65) rectangle (0.78,0.75);
%         \draw[red,ultra thick,rounded corners] (0.60,0.01) rectangle (0.75,0.99); % coordinates unten links(x,y) oben rechts(x,y)
%             \draw[blue,ultra thick,rounded corners] (0.40,0.01) rectangle (0.55,0.99); % coordinates unten links(x,y) oben rechts(x,y)
    \end{scope}
\end{tikzpicture}
 \end{center}
\end{frame}
\begin{frame}
 \frametitle{Where do we apply the isolated Track veto?}
 \begin{itemize}
  \item Isolated track veto is applied on top of isolated lepton veto
  \begin{itemize}
   \item We are interested in the fraction of lost isolated electron and muon events which are selected by the isolated track veto
  \end{itemize}
  \item Rejection efficiency derived on simulated \ttbar \& \wpj events applied to data
  \item Need to validate in data using Tag\&Probe
 \end{itemize}
 \begin{center}
 MC truth vs Tag\&Probe efficiencies technic
 \end{center}
 \begin{tabular}{|p{0.45\textwidth}|p{0.45\textwidth}|}
 \hline
  Use gen information to select objects & Select object by fitting signal \& bkg (e.g. Z mass peak) \\ \hline
  Eff.: Start with non detector object gen particle & Eff.: Start with detector object. (most basic desirable) BUT need to have reasonable ratio signal/bkg for mass peak fit\\
  \hline
 \end{tabular}
\end{frame}

\begin{frame}
 \frametitle{Isolated Track Definition}
\begin{itemize}
 \item Muon, Electron Tracks:
 \begin{itemize}
  \item Charged PFCand, $\pt>5 \gev, |\eta|<2.5$, $\mt<100 \gev$, ask for pdgID=11,13
  \item Iso: $\Sigma ( \pt\text(Tracks)\Delta R<0.3 )/(\pt Track) < 0.2$ (with $dz<0.1$)
 \end{itemize}
 \item Pion Tracks:
 \begin{itemize}
  \item Charged PFCand, $\pt>10 \gev, |\eta|<2.5$, $\mt<100 \gev$, ask for pdgID=211
  \item Iso: $\Sigma ( \pt\text(Tracks)\Delta R<0.3 )/(\pt Track) < 0.1$ (with $dz<0.1$)
 \end{itemize}
 \end{itemize}
\end{frame}
\begin{frame}
 \frametitle{Transfer from \wpj \& \ttbar to DY events}
 \begin{itemize}
  \item Need to account for kinematic differences in efficiencies parametrization: Use \pt Activity around tracks
  \begin{itemize}
   \item Problem: $\mu \& e$ tracks defined $\pt>5 \gev$ $\mu \& e$ CS only defined $\pt>10 \gev$
  \end{itemize}
  \item Isolated tracks selection: $\mt<100 \gev$ Not defined in \Zll
  \begin{itemize}
   \item Subtract tag lepton \pt from MET to emulate $\nu$
  \end{itemize}
 \end{itemize}
\begin{center}
\begin{overpic}[width=.40\textwidth]{figures/efficiencies/tagandprobe/TagAndProbe__MTW__MuIso_vs_MuIsoMTWClean__Baseline.pdf}      %\put(18,36.2){\color{red}\line(1,0){75}}
      \end{overpic}
\begin{overpic}[width=.40\textwidth]{figures/efficiencies/tagandprobe/TagAndProbe__MTW__ElecIso_vs_ElecIsoMTWClean__Baseline.pdf}      %\put(18,36.2){\color{red}\line(1,0){75}}
      \end{overpic} 
\end{center}


\end{frame}



\begin{frame}
 \frametitle{Comparison \ttbar \& \wpj vs Tag \& Probe Efficiencies}
  \begin{columns}

   \begin{column}{0.33\textwidth}
     \begin{itemize}
   \item $\mu$ track \ttbar \& \wpj eff. (truth info.)
  \end{itemize}
    \begin{tikzpicture}
    \node[anchor=south west,inner sep=0] (image) at (0,0) {\includegraphics[width=1.\textwidth]{figures/efficiencies/ttbar-wpj/MuIsoTrackGenMuReductionPTActivity.pdf}};
    \begin{scope}[x={(image.south east)},y={(image.north west)}]
%         \draw[red,ultra thick,rounded corners] (0.62,0.65) rectangle (0.78,0.75);
%         \draw[red,ultra thick,rounded corners] (0.60,0.01) rectangle (0.75,0.99); % cordinates unten links(x,y) oben rechts(x,y)
    \end{scope}
   \end{tikzpicture}
   \end{column}
   \begin{column}{0.33\textwidth}
   \begin{itemize}
    \item $\mu$ track DY eff. \\(Tag\&Probe)
   \end{itemize}

    \begin{tikzpicture}
    \node[anchor=south west,inner sep=0] (image) at (0,0) {\includegraphics[width=1.\textwidth]{figures/efficiencies/tagandprobe/MuTrackTagAndProbeMC.pdf}};
    \begin{scope}[x={(image.south east)},y={(image.north west)}]
%         \draw[red,ultra thick,rounded corners] (0.62,0.65) rectangle (0.78,0.75);
%         \draw[red,ultra thick,rounded corners] (0.60,0.01) rectangle (0.75,0.99); % cordinates unten links(x,y) oben rechts(x,y)
    \end{scope}
   \end{tikzpicture}
   \end{column}
           \begin{column}{0.33\textwidth}
   \begin{itemize}
    \item Ratio: \ttbar \& \wpj / Tag\&Probe
   \end{itemize}

    \begin{tikzpicture}
     \node[anchor=south west,inner sep=0] (image) at (0,0) {\includegraphics[width=1.\textwidth]{figures/efficiencies/tagandprobe/MuIsoTrackGenMuPTActivity_ratio.pdf}};
    \begin{scope}[x={(image.south east)},y={(image.north west)}]
%         \draw[red,ultra thick,rounded corners] (0.62,0.65) rectangle (0.78,0.75);
%         \draw[red,ultra thick,rounded corners] (0.60,0.01) rectangle (0.75,0.99); % cordinates unten links(x,y) oben rechts(x,y)
    \end{scope}
   \end{tikzpicture}
   \end{column}
  \end{columns}

\end{frame}



\begin{frame}
 \frametitle{Comparison \ttbar \& \wpj vs Tag \& Probe Efficiencies}
  \begin{columns}

   \begin{column}{0.33\textwidth}
     \begin{itemize}
   \item e track \ttbar \& \wpj eff. (truth info.)
  \end{itemize}
    \begin{tikzpicture}
    \node[anchor=south west,inner sep=0] (image) at (0,0) {\includegraphics[width=1.\textwidth]{figures/efficiencies/ttbar-wpj/ElecIsoTrackGenElecReductionPTActivity.pdf}};
    \begin{scope}[x={(image.south east)},y={(image.north west)}]
%         \draw[red,ultra thick,rounded corners] (0.62,0.65) rectangle (0.78,0.75);
%         \draw[red,ultra thick,rounded corners] (0.60,0.01) rectangle (0.75,0.99); % cordinates unten links(x,y) oben rechts(x,y)
    \end{scope}
   \end{tikzpicture}
   \end{column}
   \begin{column}{0.33\textwidth}
   \begin{itemize}
    \item e track DY eff. \\(Tag\&Probe)
   \end{itemize}

    \begin{tikzpicture}
    \node[anchor=south west,inner sep=0] (image) at (0,0) {\includegraphics[width=1.\textwidth]{figures/efficiencies/tagandprobe/ElecTrackTagAndProbeMC.pdf}};
    \begin{scope}[x={(image.south east)},y={(image.north west)}]
%         \draw[red,ultra thick,rounded corners] (0.62,0.65) rectangle (0.78,0.75);
%         \draw[red,ultra thick,rounded corners] (0.60,0.01) rectangle (0.75,0.99); % cordinates unten links(x,y) oben rechts(x,y)
    \end{scope}
   \end{tikzpicture}
   \end{column}
           \begin{column}{0.33\textwidth}
   \begin{itemize}
    \item Ratio: \ttbar \& \wpj / Tag\&Probe
   \end{itemize}

    \begin{tikzpicture}
     \node[anchor=south west,inner sep=0] (image) at (0,0) {\includegraphics[width=1.\textwidth]{figures/efficiencies/tagandprobe/ElecIsoTrackGenElecPTActivity_ratio.pdf}};
    \begin{scope}[x={(image.south east)},y={(image.north west)}]
%         \draw[red,ultra thick,rounded corners] (0.62,0.65) rectangle (0.78,0.75);
%         \draw[red,ultra thick,rounded corners] (0.60,0.01) rectangle (0.75,0.99); % cordinates unten links(x,y) oben rechts(x,y)
    \end{scope}
   \end{tikzpicture}
   \end{column}
  \end{columns}

\end{frame}




% \begin{frame}
%  \frametitle{Iso track reduction by component 1}
%  \begin{overpic}[width=.40\textwidth]{figures/efficiencies/ttbar-wpj/IsoTrackReductionPTActivity.pdf}      %\put(18,36.2){\color{red}\line(1,0){75}}
%       \end{overpic}\\
%  \begin{overpic}[width=.32\textwidth]{figures/efficiencies/ttbar-wpj/MuIsoTrackReductionPTActivity.pdf}      %\put(18,36.2){\color{red}\line(1,0){75}}
%       \end{overpic}
%  \begin{overpic}[width=.32\textwidth]{figures/efficiencies/ttbar-wpj/ElecIsoTrackReductionPTActivity.pdf}      %\put(18,36.2){\color{red}\line(1,0){75}}
%       \end{overpic}
%  \begin{overpic}[width=.32\textwidth]{figures/efficiencies/ttbar-wpj/PionIsoTrackReductionPTActivity.pdf}      %\put(18,36.2){\color{red}\line(1,0){75}}
%       \end{overpic}
% \end{frame}



\begin{frame}
 \begin{block}{}
 \centering
 \Large Backup
 \end{block}
\end{frame}





% --------------------------------------------------

\setcounter{framenumber}{32}

\end{document}

