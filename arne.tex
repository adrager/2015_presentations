\documentclass{beamer}
%kdfj
\usetheme[secheader]{Boadilla}
\setbeamertemplate{footline} {
  %\leavevmode%
  \hbox{%
  \begin{beamercolorbox}[wd=.5\paperwidth,ht=2.25ex,dp=1ex,left]{author in head/foot}%
    \usebeamerfont{author in head/foot}\hspace*{2ex}\insertshortauthor~~(adraeger@cern.ch)
  \end{beamercolorbox}%
  \begin{beamercolorbox}[wd=.5\paperwidth,ht=2.25ex,dp=1ex,right]{date in head/foot}%
    \usebeamerfont{date in head/foot}\insertshorttitle,~
    \insertshortdate{}\hspace*{1em}
    \insertframenumber{} / \inserttotalframenumber\hspace*{2ex}
  \end{beamercolorbox}}%
  \vskip0pt%
}
\beamertemplatenavigationsymbolsempty

\usepackage[percent]{overpic}
\usepackage{tikz}
%\usetikzlibrary{positioning,fit,shapes.arrows,shapes.geometric,shapes.misc,shapes.multipart,calc,shadows}
\tikzstyle{every picture}+=[remember picture]
\usepackage{booktabs}
\usepackage{graphicx}
\usepackage{rotating}
\usepackage{wasysym}
\usepackage{marvosym}
\graphicspath{{../../logo/}{figures/}{../../graphic-common/}}

\input{definitions.tex}
\newcommand{\lib}[1]{\tiny #1}

% Title etc
\vskip2cm
\title[RA2/b Meeting]{Absch\"atzung des $W$ und \ttbar Untergrundes f\"ur die Suche nach neuer Physik im Endzustand mit Jets und fehlender Transversalenergie bei CMS}
% \subtitle{Combining Single Muon and Electron Control Sample}
\author[Arne-Rasmus~Dr\"ager]{
  \underline{Arne-Rasmus~Dr\"ager}, Christian Sander \\(Uni Hamburg)
}
\date[March 05, 2015]{March 05, 2015
  \vskip1cm
  \begin{center}
    \includegraphics[height=1.5cm]{Universitaet-Hamburg-Logo.jpg}
    \hskip8cm
    \includegraphics[height=1.5cm]{CMSlogo.jpeg}
  \end{center}
}

% pdflatex packages
\hypersetup{bookmarks=true}
\hypersetup{unicode=false}
\hypersetup{pdftitle={Lost-Lepton}}
\hypersetup{pdfauthor={Arne-Rasmus~Dr\"ager}}


\begin{document}
% ==================================================
% --------------------------------------------------
\begin{frame}
  \titlepage
\end{frame}
\section{Outline}
\begin{frame}
\frametitle{Outline}
\Large
\begin{itemize}
 \item Introduction:
 \begin{itemize}
  \item SUSY
  \item Analysis strategy 8 TeV vs 13 TeV
 \end{itemize}
 \item \wpj \& \ttbar estimiation method
 \item Conclusion
\end{itemize}
\end{frame}

\section{Introduction}
\subsection{SUSY}
\begin{frame}
 \frametitle{Supersymmetry: An elegant extension of the standard model!}
 \begin{columns}
 \begin{column}{0.5\textwidth}

  \begin{itemize}
  \item SUSY advantages:
  \begin{itemize}
   \item Unification of gauge couplings possible
   \item Can provide dark matter candidate
   \item Solving hierarchy problem by canceling lare SM mass contributions to Higgs mass
  \end{itemize}
  \item New light (TeV) scale particles favored
  \begin{itemize}
   \item \textbf{Production @ LHC possible!}
  \end{itemize}
 \end{itemize}
 \end{column}
 \begin{column}{0.5\textwidth}
  \begin{overpic}[width=.90\textwidth]{figures/Sketches/CouplingsUnification.png} \end{overpic}\\
  \begin{overpic}[width=.48\textwidth]{figures/Sketches/DarkMatter.png} \end{overpic}
  \begin{overpic}[width=.48\textwidth]{figures/Sketches/HiggsMassCancelation.png} \end{overpic}
 \end{column}

 \end{columns}

\end{frame}

\subsection{Full Hadronic Analysis}

\begin{frame}
 \begin{overpic}[width=.32\textwidth]{figures/feynmanDiagrams/T1qqqq.pdf} \end{overpic}
 \begin{overpic}[width=.32\textwidth]{figures/feynmanDiagrams/T1tttt.pdf} \end{overpic}
 \begin{overpic}[width=.32\textwidth]{figures/feynmanDiagrams/T1bbbb_feyn.pdf} \end{overpic}\\
 \begin{itemize}
  \item Focus on gluino topologies inspired by natural SUSY with R-partiy conservation ($R=(-1)^{3(B-L)+2S}$)
  \item Select events with final state of only jets according to ``Baseline-Selection``:
  \begin{block}{}
  \begin{itemize}
   \item $\HT >500 \gev$ (Jets: $\pt>30\gev$, $|\eta|<2.5$)
   \item $\MHT >200 \gev$ (Jets: $\pt>30\gev$, $|\eta|<5.0$)
   \item $\NJets\ge 4$, $\BTags$= {$0,1,2,\geq3$},  \HT jets
   \item \dphin $> 4.0$ (QCD multi-jet rejection)
   \item Veto isolated $e/\mu$ (reject \ttbar \& \wpj)
  \end{itemize}
  \end{block}
  \item Increase sensitivity by dividing ``Baseline-Selecttion`` in $~72$ exclusive search bins of \HT, \MHT, \NJets \& \BTags  
 \end{itemize}
\end{frame}

\begin{frame}
 \begin{block}{}
 \centering
 \Large
 This talk: Estimation strategy of lost $e/\mu$\\ from \ttbar \& \wpj events
 \end{block}
\end{frame}
\section{Classical Lost-Lepton Method}
\begin{frame}
 \begin{center}
\frametitle{Mainly \ttbar \& \wpj events where prompt electrons and muons are lost}

 \begin{overpic}[width=0.7\textwidth]{figures/Sketches/TotalLostLeptonBackgroundSmaller.png} 
%    \put(0,0){\rotatebox{-0}{\normalsize  Select a control-sample (CS) of exaclty one well isolated $e/\mu$ within acceptance cuts}}
%    \put(0,-10){\rotatebox{-0}{\normalsize  Weight each CS event according to efficiencies for each identification step}}
 \end{overpic}

 \end{center}
 \begin{itemize}
  \item Select a control-sample (CS) of exaclty one well isolated $e/\mu$ within acceptance cuts
  \item Weight each CS event according to efficiencies for each identification step
 \end{itemize}

\end{frame}

\begin{frame}
\begin{itemize}
 \item Signal and other SM processes can contribute to e/$\mu$ control sample
 \item Suppress contamination by requiring trans. mass $\mt < 100 \gev$ \\
\end{itemize}
\vspace{0.5cm}
\hspace{0.5cm}$m_{T} = \sqrt{2 \cdot p_{T}(\mu/e)\cdot \met (1 - \cos(\Delta \Phi))}$

  \begin{columns}
    \begin{column}{0.5\textwidth}

      \begin{itemize}
      \item Removes about 10\% of e/$\mu$ CS due to:
        \begin{itemize}
        \item di-leptonic \ttbar decays
        \item Mismeasured jets
        \item Highly virtual W
        \end{itemize}
      \begin{centering}
      \end{centering}
      \item Correction for as a function of lepton $p_{T}$ \& activity around the lepton (motivation \& definition on the next slides)
      \end{itemize}
      \vspace{0.3cm}
    \end{column}
    \begin{column}{0.5\textwidth}
      \centering
       \begin{overpic}[width=0.95\textwidth]{figures/control-sample/mtw.png}
       \put(53.93,80){\color{black}\line(0,-1){67}}
       \end{overpic}
    \end{column}
  \end{columns}
\end{frame}


\begin{frame}
\frametitle{Comparison prediction to expectation on simulated events}
  \begin{columns}
    \begin{column}{0.5\textwidth}
     \centering
      \begin{overpic}[width=0.57\textwidth]{figures/lost-lepton-closure/Closure_Combined__HT__MCEx_vs_MuPrMTWDiLep+ElecPrMTWDiLep__Baseline.pdf}
     \end{overpic}
           \begin{overpic}[width=0.57\textwidth]{figures/lost-lepton-closure/Closure_Combined__MHT__MCEx_vs_MuPrMTWDiLep+ElecPrMTWDiLep__Baseline.pdf}
     \end{overpic}
    \end{column}
    \begin{column}{0.5\textwidth}
      \centering
            \begin{overpic}[width=0.57\textwidth]{figures/lost-lepton-closure/Closure_Combined__NJets__MCEx_vs_MuPrMTWDiLep+ElecPrMTWDiLep__Baseline.pdf}
     %     \put(90,90){\rotatebox{-45}{\scriptsize \Large Arne}}
     \end{overpic}
     \begin{overpic}[width=0.57\textwidth]{figures/lost-lepton-closure/Closure_Combined__BTags__MCEx_vs_MuPrMTWDiLep+ElecPrMTWDiLep__Baseline.pdf}
%       \begin{overpic}[width=0.57\textwidth]{figures/lost-lepton/Closure__NVtx__MCPrMTWDiLep_vs_MCEx__csa_Baseline.pdf}
      \end{overpic}
    \end{column}
  \end{columns}
  \begin{itemize}
  \item $\HT > 500 \gev$ , $\MHT>200 \gev$, $\NJets\ge 3$ , \dphin $> 4.0$
   \item Reasonable closure (exp $4352.1\pm 20.0$, pred $4432.4 \pm 32.0$)
  \end{itemize}
\end{frame}



\begin{frame}
% \frametitle{Lost-Lepton Background $\mu$ \& e}
 \begin{center}
 \begin{overpic}[width=0.80\textwidth]{figures/Sketches/LostLeptonSketch_mu_pred_full.pdf} 
  \put(0,-3){\rotatebox{-0}{\normalsize Expectation \& Prediction using single $\mu$ control-sample (CS)}}
 \end{overpic}

 \end{center}
\end{frame}


\begin{frame}
\frametitle{Lost-Lepton Background $\mu$ only}
 \begin{center}
 \begin{overpic}[width=1.0\textwidth]{figures/Sketches/LostLeptonSketch.pdf} \end{overpic}
 \end{center}
\end{frame}

\begin{frame}
\frametitle{Lost-Lepton Background $\mu$ only}
 \begin{center}
 \begin{overpic}[width=0.9\textwidth]{figures/Sketches/LostLeptonSketch_mu_pred.pdf} 
 \put(0,10){\rotatebox{-0}{\normalsize Expectation \& Prediction using single $\mu$ control-sample (CS)}}
 \end{overpic}

 \end{center}
\end{frame}


\begin{frame}
\frametitle{Lost-Lepton Background $\mu$ \& e}
 \begin{center}
 \begin{overpic}[width=0.9\textwidth]{figures/Sketches/LostLeptonSketch_ll.pdf} 
%  \put(0,10){\rotatebox{-0}{\normalsize Expectation \& Prediction using single $\mu$ control sample (CS)}}
 \end{overpic}

 \end{center}
\end{frame}


\begin{frame}
\frametitle{Lost-Lepton Background $\mu$ \& e}
 \begin{center}
 \begin{overpic}[width=0.80\textwidth]{figures/Sketches/LostLeptonSketch_mu_pred_full.pdf} 
  \put(0,-3){\rotatebox{-0}{\normalsize Expectation \& Prediction using single $\mu$ control-sample (CS)}}
 \end{overpic}

 \end{center}
\end{frame}

\begin{frame}
\frametitle{Lost-Lepton Background $\mu$ \& e}
 \begin{center}
 \begin{overpic}[width=0.80\textwidth]{figures/Sketches/LostLeptonSketch_e_pred_full.pdf} 
  \put(0,-3){\rotatebox{-0}{\normalsize Expectation \& Prediction using single e control-sample (CS)}}
 \end{overpic}

 \end{center}
\end{frame}
\begin{frame}
  \begin{center}
    \Large
     Backup
  \end{center}
\end{frame}


\begin{frame}
\frametitle{Baseline selection}
\normalsize
\begin{itemize}
 \item $\HT >500 \gev$
 \begin{itemize}
       \item Jets: $\pt>30\gev$, $|\eta|<2.5$ 
      \end{itemize}
 \item $\MHT >200 \gev$
  \begin{itemize}
       \item Jets: $\pt>30\gev$, $|\eta|<5.0$
      \end{itemize}
 \item $\NJets\ge 4$, \HT jets
 \item $\BTags$= {$0,1,2,\geq3$} CSVM ($>0.814$), $\pt>30\gev$
% \item $\deltaphi_{1,2,3}>0.5,0.5,0.3$
\item  \dphin $> 4.0$
\item Veto Muons: \href{https://twiki.cern.ch/twiki/bin/view/CMSPublic/SWGuideMuonId\#Tight\_Muon}{2012 ``tight'' ID}: $p_T > 10$ GeV, $I_{rel}\; (\Delta R<0.4) < 0.2$    
    \item Veto Electrons: \href{https://twiki.cern.ch/twiki/bin/viewauth/CMS/CutBasedElectronIdentificationRun2\#CSA14\_selection\_conditions\_25ns}{Phys14 POG ID}:  $p_T > 10$ GeV, $I_{rel}\;
      (\Delta R<0.3) < 0.33 / 0.38$
      \item Filters:
      \begin{itemize}
       \item Lose jet ID criteria for pf jets
      \end{itemize}

%       \item Under study:
%       \begin{itemize}
% 
% 
%     \item Taus: \href{https://indico.cern.ch/event/359233/contribution/4/material/slides/0.pdf}{Phys14 POG ID}: $p_T > 10$ GeV, $|\eta| < 2.3$,
%       chargedIsoPtSum $(\Delta R<0.5)$ < 1.0 GeV (no neutral isolation yet)
%     \item Isolated tracks: $p_T > 15$ GeV, $I_{rel}\;(\Delta R<0.3) < 0.1$ -- just charged candidates
%       \end{itemize}
\end{itemize}
% \begin{block}{}
% \centering
% \Large
% \end{block}
\end{frame}

% --------------------------------------------------

\setcounter{framenumber}{7}

\end{document}
