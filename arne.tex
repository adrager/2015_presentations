\documentclass{beamer}
%kdfj
\usetheme[secheader]{Boadilla}
\setbeamertemplate{footline} {
  %\leavevmode%
  \hbox{%
  \begin{beamercolorbox}[wd=.5\paperwidth,ht=2.25ex,dp=1ex,left]{author in head/foot}%
    \usebeamerfont{author in head/foot}\hspace*{2ex}\insertshortauthor~~(adraeger@cern.ch)
  \end{beamercolorbox}%
  \begin{beamercolorbox}[wd=.5\paperwidth,ht=2.25ex,dp=1ex,right]{date in head/foot}%
    \usebeamerfont{date in head/foot}\insertshorttitle,~
    \insertshortdate{}\hspace*{1em}
    \insertframenumber{} / \inserttotalframenumber\hspace*{2ex}
  \end{beamercolorbox}}%
  \vskip0pt%
}
\beamertemplatenavigationsymbolsempty

\usepackage[percent]{overpic}
\usepackage{tikz}
%\usetikzlibrary{positioning,fit,shapes.arrows,shapes.geometric,shapes.misc,shapes.multipart,calc,shadows}
\tikzstyle{every picture}+=[remember picture]
\usepackage{booktabs}
\usepackage{graphicx}
\usepackage{rotating}
\usepackage{wasysym}
\usepackage{marvosym}
\usepackage{amssymb}
\usepackage{xcolor}
\graphicspath{{../../logo/}{figures/}{../../graphic-common/}}

\input{definitions.tex}
\newcommand{\lib}[1]{\tiny #1}

% Title etc
\vskip2cm
\title[RA2/b Meeting]{Classical Lost-Lepton Background}
\subtitle{Study of Mini Isolation\\Usage of $e$ \& $\mu$ Prediction} 
\author[Arne-Rasmus~Dr\"ager]{
  Arne-Rasmus~Dr\"ager(Uni Hamburg)
}
\date[March 03, 2015]{March 03, 2015
  \vskip1cm
  \begin{center}
    \includegraphics[height=1.5cm]{Universitaet-Hamburg-Logo.jpg}
    \hskip8cm
    \includegraphics[height=1.5cm]{CMSlogo.jpeg}
  \end{center}
}

% pdflatex packages
\hypersetup{bookmarks=true}
\hypersetup{unicode=false}
\hypersetup{pdftitle={Lost-Lepton}}
\hypersetup{pdfauthor={Arne-Rasmus~Dr\"ager}}


\begin{document}
% ==================================================
% --------------------------------------------------
\begin{frame}
  \titlepage
\end{frame}


\section{Lost-Lepton Method}
\subsection{Overview}
\begin{frame}
 \frametitle{Total Lost-Lepton Background: Lost $\mu$ \& $e$}

 \begin{center}
 \begin{overpic}[width=0.9\textwidth]{figures/Sketches/LostLeptonSketch_ll.pdf} 
%  \put(0,10){\rotatebox{-0}{\normalsize Expectation \& Prediction using single $\mu$ control sample (CS)}}
 \end{overpic}

 \end{center}

\end{frame}

\begin{frame}

 \begin{center}
 \begin{overpic}[width=0.75\textwidth]{figures/Sketches/LostLeptonSketch_mu_pred_full.pdf} 
%  \put(0,10){\rotatebox{-0}{\normalsize Expectation \& Prediction using single $\mu$ control sample (CS)}}
 \end{overpic}

 \end{center}
 \begin{itemize}
  \item Select a control-sample (CS) of exactly one well isolated e, $\mu$ within the acceptance
  \item Weight each event according to efficiencies of each identification step
 \end{itemize}
\end{frame}
\subsection{Usage of $\mu$ \& $e$ CS}
\begin{frame}
 \frametitle{Making the most of $\mu$ \& $e$ CS}
 \begin{itemize}
  \item Method has been extend to use both $\mu$ \& $e$ CS to predict lost $\mu$ \& $e$ events
  \item Proposals:
  \begin{itemize}
   \item Treat $\mu$ and  $e$ events totally separated: Use only $\mu$ CS to predict lost $\mu$ \& only $e$ CS to predict lost $e$
   \item Take advantage of both CS for both backgrounds: Use $\mu$ ($e$) CS to predict both $\mu$ \& $e$. Combine according to statistical power
  \end{itemize}
  \item What are the advantages of each method?
  \item Is one approach superior in terms of the statistical uncertainties of total lost-lepton prediction?
  \item What is the impact of different $\mu$ \& $e$ control-sample sizes?

 \end{itemize}

\end{frame}
\begin{frame}
 \frametitle{Comparison of $\mu$ \& $e$ Control-Sample (Search Bins)}
%  \begin{overpic}[width=0.47\textwidth]{figures/control-sample/ControlSampleNumbersCompare.pdf} 
%  \put(73,55){\rotatebox{-0}{\scriptsize \mycirc[blue] $\mu$ CS}}
%  \put(73,50){\rotatebox{-0}{\scriptsize \mycirc[red] $e$ CS}}
%  \end{overpic}
\begin{center}
 


 \begin{overpic}[width=0.45\textwidth]{figures/classicIso/control-sample/Closure_SearchBins_Separated__Bin__MuCS_vs_ElecCS__Search_Bins.pdf} 
%  \put(0,10){\rotatebox{-0}{\normalsize Expectation \& Prediction using single $\mu$ control sample (CS)}}
 \end{overpic}
 \end{center}
 \begin{itemize}
  \item Ratio of $\mu$/ $e$ CS mostly 1. But especially in low statistic regions fluctuations expected
 \end{itemize}

\end{frame}

\begin{frame}
  \frametitle{Comparison of both approaches: Simple 1 bin Example}
  \begin{itemize}
   \item Assume:
   \begin{itemize}
    \item Both methods predict same amount of lost-leptons ($\mu$ and $e$)
    \item Control-samples: 4 $\mu$, 1 $e$ (weight=1)
    \item Same weight for each $\mu$ \& $e$ 
   \end{itemize}

  \end{itemize}
  \begin{columns}
   \begin{column}{0.5\textwidth}
   \begin{center}
    Separate prediction
   \end{center}

   \end{column}
   \begin{column}{0.5\textwidth}
    \begin{center}
    Combined prediction
    \end{center}
   \end{column}

  \end{columns}

  \scriptsize
  \begin{tabular}{l|r|r||r|r}

                                                  &           $\mu$            &           $e$  &             $\mu$            &           $e$  \\
\midrule 
     CS &                $4\pm2$ &             $1\pm1$ &              $4\pm2$ &             $1\pm1$  \\ \hline
      Pred.    &          $10\pm5$ (50\%) &              $10\pm10$(100\%)&              $20\pm10$(50\%)&                 $20\pm20$(100\%) \\
      Total Pred. &      $10\pm5$ (50\%) &              $10\pm10$(100\%)&              $20\pm10$(50\%)&                 $20\pm20$(100\%) \\
\bottomrule
\end{tabular}
\small
\begin{itemize}
 \item Total lost-lepton prediction:
 \item Treating $\mu$ \& $e$ separately
 \begin{itemize}
  \item Total prediction: $20\pm11.18$ (quadratic combination of stat. independent prediction uncertainties)
 \end{itemize}
 \item Using $\mu$ \& $e$ for both lost $\mu$ \& $e$
 \begin{itemize}
  \item Total prediction: $20\pm8.95$ (using weighted average)
 \end{itemize}
\end{itemize}
\end{frame}

\section{Mini Isolation VS Classical Isolation}
\begin{frame}
 \frametitle{Mini Isolation (following UCSB approach \href{https://indico.cern.ch/event/368826/contribution/3/material/slides/0.pdf}{Adam talk})}
   \begin{columns}
   \begin{column}{0.65\textwidth}
 \begin{itemize}
  \item Variable isolation cone $R$ depending on lepton \pt 
  \item Chose $R$ small as possible to reduce overlap with jets in boosted systems
  \item Keep large enough to not pick up b-decay leptons
  
  \item Checked working point with:
  \begin{itemize}
   \item Max isolation cone $R = 0.2$ ($\pt,lep \leq50 GeV$)
   \item Med isolation cone $R=\frac{10 GeV}{\pt,lep}$ ($50 GeV \geq\pt,lep \leq200 GeV$)
   \item Min isolation cone $R = 0.05$ ($\pt,lep \geq200 GeV$)
  \end{itemize}
 \end{itemize}
 \end{column}
 \begin{column}{0.35\textwidth}
 \vskip3cm
  \begin{overpic}[width=1.0\textwidth]{figures/Sketches/miniIso.png} 
%  \put(0,10){\rotatebox{-0}{\normalsize Expectation \& Prediction using single $\mu$ control sample (CS)}}
 \end{overpic}
 \end{column}
 \end{columns}

\end{frame}

\begin{frame}
 \frametitle{Comparison of isolation efficiencies}
 
  \begin{columns}
   \begin{column}{0.5\textwidth}
   \begin{center}
    Old Isolation \\
    \begin{overpic}[width=0.90\textwidth]{figures/classicIso/efficiencies/MuIsoClassicLepHT.pdf} 
%  \put(0,10){\rotatebox{-0}{\normalsize Expectation \& Prediction using single $\mu$ control sample (CS)}}
 \end{overpic}\\
   \begin{overpic}[width=0.90\textwidth]{figures/classicIso/efficiencies/ElecIsoHTEff.pdf} 
%  \put(0,10){\rotatebox{-0}{\normalsize Expectation \& Prediction using single $\mu$ control sample (CS)}}
 \end{overpic}
   \end{center}

   \end{column}
   \begin{column}{0.5\textwidth}
    \begin{center}
    Mini Isolation \\
      \begin{overpic}[width=0.90\textwidth]{figures/miniIsolation/efficiencies/MuIsoMiniIsoLepHT.pdf} 
%  \put(0,10){\rotatebox{-0}{\normalsize Expectation \& Prediction using single $\mu$ control sample (CS)}}
 \end{overpic} \\
   \begin{overpic}[width=0.90\textwidth]{figures/miniIsolation/efficiencies/ElecIsoHTEff.pdf} 
%  \put(0,10){\rotatebox{-0}{\normalsize Expectation \& Prediction using single $\mu$ control sample (CS)}}
 \end{overpic}
    \end{center}
   \end{column}

  \end{columns}
  \begin{itemize}
   \item Huge Improvement especially visible vs \HT
  \end{itemize}

\end{frame}



\begin{frame}
 \frametitle{Comparison of composition of expectation}
 
  \begin{columns}
   \begin{column}{0.5\textwidth}
   \begin{center}
    Old Isolation \\
 \begin{overpic}[width=0.55\textwidth]{figures/classicIso/expectation/Expectation_SearchBins_Separated__Bin__MuExIso+MuExReco+MuExAcc__Search_Bins.pdf} 
%  \put(0,10){\rotatebox{-0}{\normalsize Expectation \& Prediction using single $\mu$ control sample (CS)}}
 \end{overpic}\\
  \begin{overpic}[width=0.55\textwidth]{figures/classicIso/expectation/Expectation_SearchBins_Separated__Bin__ElecExIso+ElecExReco+ElecExAcc__Search_Bins.pdf} 
%  \put(0,10){\rotatebox{-0}{\normalsize Expectation \& Prediction using single $\mu$ control sample (CS)}}
 \end{overpic}
   \end{center}

   \end{column}
   \begin{column}{0.5\textwidth}
    \begin{center}
    Mini Isolation \\
  \begin{overpic}[width=0.55\textwidth]{figures/miniIsolation/expectation/Expectation_SearchBins_Separated__Bin__MuExIso+MuExReco+MuExAcc__Search_Bins.pdf} 
%  \put(0,10){\rotatebox{-0}{\normalsize Expectation \& Prediction using single $\mu$ control sample (CS)}}
 \end{overpic} \\
  \begin{overpic}[width=0.55\textwidth]{figures/miniIsolation/expectation/Expectation_SearchBins_Separated__Bin__ElecExIso+ElecExReco+ElecExAcc__Search_Bins.pdf} 
%  \put(0,10){\rotatebox{-0}{\normalsize Expectation \& Prediction using single $\mu$ control sample (CS)}}
 \end{overpic}
    \end{center}
   \end{column}

  \end{columns}
  \begin{itemize}
   \item Huge reduction of expected lost-lepton due to non-isolated! 
  \end{itemize}

\end{frame}


\begin{frame}
  \frametitle{Comparison of $e \& \mu$ control-sample}
 
  \begin{columns}
   \begin{column}{0.5\textwidth}
   \begin{center}
 $\mu$ control sample comparison \\
  \begin{overpic}[width=0.90\textwidth]{figures/compare/Closure_SearchBins_Separated__Bin__MuCS_vs_MuCSMiniIso__Search_Bins.pdf} 

%  \put(0,10){\rotatebox{-0}{\normalsize Expectation \& Prediction using single $\mu$ control sample (CS)}}
 \end{overpic}
   \end{center}

   \end{column}
   \begin{column}{0.5\textwidth}
    \begin{center}
 $e$ control sample comparison \\
 \begin{overpic}[width=0.90\textwidth]{figures/compare/Closure_SearchBins_Separated__Bin__ElecCS_vs_ElecCSMiniIso__Search_Bins.pdf} 
%  \put(0,10){\rotatebox{-0}{\normalsize Expectation \& Prediction using single $\mu$ control sample (CS)}}
 \end{overpic} 
    \end{center}
   \end{column}

  \end{columns}
  \begin{itemize}
   \item Over all about 10\% gain in terms of $e\& \mu$ control-sample
   \item High \HT bins gain about 25\%!
  \end{itemize}
\end{frame}

\begin{frame}
 \frametitle{Sensitivity (numbers produced by Jack. Thank you!)}

\small Baseline selection applied except lepton veto (including isolated track veto)
 \Tiny
 \begin{center}
 \Tiny
T1tttt (1500, 100) Preselection: $\BTags\geq2,\;\HT>1000,\;\NJets\geq9,\;\MHT>500$

 \begin{tabular}{lc|cc|cc}
    \hline
    \hline
    Cut & T1tttt ($m_{\gluino}=1500$ GeV, $m_{\lsp}=100$ GeV) ($0-l)$ & SM $(0-l)$ & SM $(1-l)$ & $\tau$ & $Z_{Bi}(\tau)$ \\ \hline
Old lepton veto & $6.1 \pm 0.1$ & $2.1 \pm 0.4$ & $2.7 \pm 0.5$ &  $1.3$ & $1.8$  \\
Mini isolation veto & $5.1 \pm 0.1$ & $1.9 \pm 0.4$ & $2.7 \pm 0.5$ &  $1.4$ & $1.6$  \\
    \hline
    \hline
  \end{tabular}
  \end{center}
  
  \begin{center}
 \Tiny T1tttt (1200, 800) Preselection: $\BTags\geq2,\;\NJets\geq9$
 \begin{tabular}{lc|cc|cc}
    \hline
    \hline
    Cut & T1tttt ($m_{\gluino}=1200$ GeV, $m_{\lsp}=800$ GeV) ($0-l)$ & SM $(0-l)$ & SM $(1-l)$ & $\tau$ & $Z_{Bi}(\tau)$ \\ \hline
Old lepton veto & $24.3 \pm 0.3$ & $86.1 \pm 3.7$ & $117.2 \pm 3.7$ &  $1.4$ & $1.8$  \\
Mini isolation veto & $21.1 \pm 0.3$ & $80.4 \pm 3.6$ & $125.3 \pm 3.8$ &  $1.6$ & $1.7$  \\
    \hline
    \hline
  \end{tabular}
   \end{center}
   
  \begin{center}
 \Tiny T1bbbb (1500, 100) Preselection: $\BTags\geq2,\;\HT>1400,\;\MHT>500$

  \begin{tabular}{lc|cc|cc}
    \hline
    \hline
    Cut & T1bbbb ($m_{\gluino}=1500$ GeV, $m_{\lsp}=100$ GeV) ($0-l)$ & SM $(0-l)$ & SM $(1-l)$ &  $\tau$ & $Z_{Bi}(\tau)$ \\ \hline
Old lepton veto & $14.3 \pm 0.1$ & $9.7 \pm 0.9$ & $9.3 \pm 1.0$ &  $1.0$ & $2.3$  \\
Mini isolation veto & $14.2 \pm 0.1$ & $7.9 \pm 0.8$ & $10.7 \pm 1.1$ &  $1.4$ & $2.7$  \\
    \hline
    \hline
  \end{tabular}
 \end{center}
  \begin{center}
 \Tiny T1bbbb (1000, 900) Preselection: $\BTags\geq2,\;\MHT>500$

   \begin{tabular}{lc|cc|cc}
    \hline
    \hline
    Cut & T1bbbb ($m_{\gluino}=1000$ GeV, $m_{\lsp}=900$ GeV) ($0-l)$ & SM $(0-l)$ & SM $(1-l)$ & $\tau$ & $Z_{Bi}(\tau)$ \\ \hline
Old lepton veto & $34.1 \pm 0.7$ & $56.5 \pm 2.2$ & $57.6 \pm 2.4$ &  $1.0$ & $2.8$  \\
Mini isolation veto & $33.7 \pm 0.7$ & $51.7 \pm 2.0$ & $61.7 \pm 2.5$ &  $1.2$ & $2.9$  \\
    \hline
    \hline
  \end{tabular}
  \end{center}
  With: $\tau=\frac{\text{N}_{CS}}{\text{N}_{SM}\text{exp}}$ \& $Z_{Bi}(\tau)$ expected discovery significance (sigmas) using $\text{N}_{CS}, \text{N}_{SM}\text{exp}$ \& $\tau$
\end{frame}

\begin{frame}
 \frametitle{Conclusion}
 \begin{itemize}
  \item Mini Isolation:
 \begin{itemize}
  \item Mini isolation increases the isolation efficiency at high \HT regions (increase from 70\% to 90\%!)
  \item Overall a reduction of the background arising from non-isolated $\mu$ factor of 4 and for 2.6  for $e$
  \item \wpj and \ttbar events are more efficiently rejected
  \item Increase of single $e \& \mu$ control-sample up to 25\% (over all 10\%) 
  \item No change visible for purity of control-sample
  \item No strong gain in terms of sensitivity in the most sensitive bins for t1bbbb: reduction for t1tttt
  \end{itemize}
  \item Combination of $e \& \mu$ predictions
  \begin{itemize}
   \item Comparison of the case of using $e$ cs to predict lost $e$ background and $\mu$ to predict lost $\mu$ introduces a larger uncertainty (stat. only) than using both to predict both
   \item Weighted averaging over the predictions decreases fluctuations in low stat bins
  \end{itemize}

 \end{itemize}

\end{frame}

\begin{frame}
 \begin{block}{}
 \centering
 \Large Backup
 \end{block}
\end{frame}

\begin{frame}
 \frametitle{Sensitivity (numbers produced by Jack. Thank you!)}

\small Baseline selection applied except lepton veto (including isolated track veto)
 \Tiny
 \begin{center}
 \Tiny
T1tttt (1500, 100) Preselection: $\BTags\geq2,\;\HT>1000,\;\NJets\geq9,\;\MHT>500$

  \begin{tabular}{lcccc}
    \hline
    \hline
    Cut & T1tttt ($m_{\gluino}=1500$ GeV, $m_{\lsp}=100$ GeV) & SM & $S/\sqrt{B}$ & $Z_{Bi}$ \\ \hline
No lepton veto & $10.1 \pm 0.1$ & $2.5 \pm 0.4$ &  $6.3$ & $2.4$  \\\hline
Old lepton veto & $6.1 \pm 0.1$ & $2.1 \pm 0.4$ &  $4.3$ & $1.6$  \\
Mini isolation veto & $5.1 \pm 0.1$ & $1.9 \pm 0.4$ &  $3.7$ & $1.4$  \\
    \hline
    \hline
  \end{tabular}
  \end{center}
  
  \begin{center}
 \Tiny T1tttt (1200, 800) Preselection: $\BTags\geq2,\;\NJets\geq9$

 \begin{tabular}{lcccc}
    \hline
    \hline
    Cut & T1tttt ($m_{\gluino}=1200$ GeV, $m_{\lsp}=800$ GeV) & SM & $S/\sqrt{B}$ & $Z_{Bi}$ \\ \hline
No lepton veto & $37.8 \pm 0.4$ & $110.7 \pm 4.0$ &  $3.6$ & $2.3$  \\\hline
Old lepton veto & $24.3 \pm 0.3$ & $86.1 \pm 3.7$ &  $2.6$ & $1.7$  \\
Mini isolation veto & $21.1 \pm 0.3$ & $80.4 \pm 3.6$ &  $2.4$ & $1.5$  \\
    \hline
    \hline
  \end{tabular}
   \end{center}
   
  \begin{center}
 \Tiny T1bbbb (1500, 100) Preselection: $\BTags\geq2,\;\HT>1400,\;\MHT>500$

 \begin{tabular}{lcccc}
    \hline
    \hline
    Cut & T1bbbb ($m_{\gluino}=1500$ GeV, $m_{\lsp}=100$ GeV) & SM &  $S/\sqrt{B}$ & $Z_{Bi}$ \\ \hline
No lepton veto & $14.4 \pm 0.1$ & $12.4 \pm 1.0$ &  $4.1$ & $2.2$  \\\hline
Old lepton veto & $14.3 \pm 0.1$ & $9.7 \pm 0.9$ &  $4.6$ & $2.3$  \\
Mini isolation veto & $14.2 \pm 0.1$ & $7.9 \pm 0.8$ &  $5.1$ & $2.4$  \\
    \hline
    \hline
  \end{tabular}
 \end{center}
  \begin{center}
 \Tiny T1bbbb (1000, 900) Preselection: $\BTags\geq2,\;\MHT>500$


  \begin{tabular}{lcccc}
    \hline
    \hline
    Cut & T1bbbb ($m_{\gluino}=1000$ GeV, $m_{\lsp}=900$ GeV) & SM & $S/\sqrt{B}$ & $Z_{Bi}$ \\ \hline
No lepton veto & $34.8 \pm 0.7$ & $68.6 \pm 2.4$ &  $4.2$ & $2.6$  \\\hline
Old lepton veto & $34.1 \pm 0.7$ & $56.5 \pm 2.2$ &  $4.5$ & $2.7$  \\
Mini isolation veto & $33.7 \pm 0.7$ & $51.7 \pm 2.0$ &  $4.7$ & $2.8$  \\
    \hline
    \hline
  \end{tabular}
  \end{center}
  
\end{frame}



\begin{frame}
 \frametitle{Purity comparison}
 
  \begin{columns}
   \begin{column}{0.5\textwidth}
   \begin{center}
    Old Isolation \\
 \begin{overpic}[width=0.75\textwidth]{figures/classicIso/control-sample/MuPurityClassicLepHT.pdf} 
%  \put(0,10){\rotatebox{-0}{\normalsize Expectation \& Prediction using single $\mu$ control sample (CS)}}
 \end{overpic}\\
  \begin{overpic}[width=0.75\textwidth]{figures/classicIso/control-sample/ElecPurityClassicLepHT.pdf} 
%  \put(0,10){\rotatebox{-0}{\normalsize Expectation \& Prediction using single $\mu$ control sample (CS)}}
 \end{overpic}
   \end{center}

   \end{column}
   \begin{column}{0.5\textwidth}
    \begin{center}
    Mini Isolation \\
  \begin{overpic}[width=0.75\textwidth]{figures/miniIsolation/control-sample/MuPurityMiniIsoHT.pdf} 
%  \put(0,10){\rotatebox{-0}{\normalsize Expectation \& Prediction using single $\mu$ control sample (CS)}}
 \end{overpic} \\
   \begin{overpic}[width=0.75\textwidth]{figures/miniIsolation/control-sample/ElecPurityMiniIsoHT.pdf} 
%  \put(0,10){\rotatebox{-0}{\normalsize Expectation \& Prediction using single $\mu$ control sample (CS)}}
 \end{overpic}
    \end{center}
   \end{column}

  \end{columns}
  \begin{itemize}
   \item No significant change observed
  \end{itemize}

\end{frame}



% --------------------------------------------------

\setcounter{framenumber}{12}

\end{document}

