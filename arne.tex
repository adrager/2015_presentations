\documentclass{beamer}
%kdfj
\usetheme[secheader]{Boadilla}
\setbeamertemplate{footline} {
  %\leavevmode%
  \hbox{%
  \begin{beamercolorbox}[wd=.5\paperwidth,ht=2.25ex,dp=1ex,left]{author in head/foot}%
    \usebeamerfont{author in head/foot}\hspace*{2ex}\insertshortauthor~~(adraeger@cern.ch)
  \end{beamercolorbox}%
  \begin{beamercolorbox}[wd=.5\paperwidth,ht=2.25ex,dp=1ex,right]{date in head/foot}%
    \usebeamerfont{date in head/foot}\insertshorttitle,~
    \insertshortdate{}\hspace*{1em}
    \insertframenumber{} / \inserttotalframenumber\hspace*{2ex}
  \end{beamercolorbox}}%
  \vskip0pt%
}
\beamertemplatenavigationsymbolsempty

\usepackage[percent]{overpic}
\usepackage{tikz}
%\usetikzlibrary{positioning,fit,shapes.arrows,shapes.geometric,shapes.misc,shapes.multipart,calc,shadows}
\tikzstyle{every picture}+=[remember picture]
\usepackage{booktabs}
\usepackage{graphicx}
\usepackage{rotating}
\usepackage{wasysym}
\usepackage{marvosym}
\usepackage{amssymb}
\usepackage{xcolor}
\graphicspath{{../../logo/}{figures/}{../../graphic-common/}}

\input{definitions.tex}
\newcommand{\lib}[1]{\tiny #1}

% Title etc
\vskip2cm
\title[RA2/b Meeting]{Classical Lost-Lepton Background}
\subtitle{Study of Mini Isolation\\Usage of $e$ \& $\mu$ Prediction\\Stat. Uncertainty on Efficiencies} 
\author[Arne-Rasmus~Dr\"ager]{
  Arne-Rasmus~Dr\"ager(Uni Hamburg)
}
\date[March 05, 2015]{March 05, 2015
  \vskip1cm
  \begin{center}
    \includegraphics[height=1.5cm]{Universitaet-Hamburg-Logo.jpg}
    \hskip8cm
    \includegraphics[height=1.5cm]{CMSlogo.jpeg}
  \end{center}
}

% pdflatex packages
\hypersetup{bookmarks=true}
\hypersetup{unicode=false}
\hypersetup{pdftitle={Lost-Lepton}}
\hypersetup{pdfauthor={Arne-Rasmus~Dr\"ager}}


\begin{document}
% ==================================================
% --------------------------------------------------
\begin{frame}
  \titlepage
\end{frame}
\section{Outline}
\begin{frame}
\frametitle{Outline}
\Large
\begin{itemize}
 \item Classical Lost-Lepton Method (reminder)
 \item Usage of $e$ \& $\mu$ Control-Sample
 \item Stat. Uncertainty on Efficiencies
 \item Study of Mini Isolation
 \item Conclusion
\end{itemize}
\end{frame}

\section{Lost-Lepton Method}
\subsection{Overview}
\begin{frame}
 \frametitle{Total Lost-Lepton Background: Lost $\mu$ \& $e$}

 \begin{center}
 \begin{overpic}[width=0.9\textwidth]{figures/Sketches/LostLeptonSketch_ll.pdf} 
%  \put(0,10){\rotatebox{-0}{\normalsize Expectation \& Prediction using single $\mu$ control sample (CS)}}
 \end{overpic}

 \end{center}

\end{frame}

\begin{frame}

 \begin{center}
 \begin{overpic}[width=0.75\textwidth]{figures/Sketches/LostLeptonSketch_mu_pred_full.pdf} 
%  \put(0,10){\rotatebox{-0}{\normalsize Expectation \& Prediction using single $\mu$ control sample (CS)}}
 \end{overpic}

 \end{center}
 \begin{itemize}
  \item Select a control-sample (CS) of exaclty one well isolated e, $\mu$ within the acceptance
  \item Weight each event according to efficiencies of each identificaiton step
 \end{itemize}
\end{frame}
\subsection{Usage of $\mu$ \& $e$ CS}
\begin{frame}
 \frametitle{Making the most of $\mu$ \& $e$ CS}
 \begin{itemize}
  \item Method as been extend to use both $\mu$ \& $e$ CS to predict lost $\mu$ \& $e$ events
  \item Proposals:
  \begin{itemize}
   \item Treat $\mu$ and  $e$ events totally separated: Use only $\mu$ CS to predict lost $\mu$ \& only $e$ CS to predict lost $e$
   \item Take advantage of both CS for both backgrounds: Use $\mu$ ($e$) CS to predict both $\mu$ \& $e$. Combine according to statistical power
  \end{itemize}
  \item What are the advantages of each method?
  \item Is one approach superior in terms of the statistical uncertaitnies of total lost-lepton prediction?
  \item What is the impact of different $\mu$ \& $e$ control-sample sizes?

 \end{itemize}

\end{frame}
\begin{frame}
 \frametitle{Comparison of $\mu$ \& $e$ Control-Sample (Search Bins)}
 \begin{overpic}[width=0.47\textwidth]{figures/control-sample/ControlSampleNumbersCompare.pdf} 
 \put(73,55){\rotatebox{-0}{\scriptsize \mycirc[blue] $\mu$ CS}}
 \put(73,50){\rotatebox{-0}{\scriptsize \mycirc[red] $e$ CS}}
 \end{overpic}
 \begin{overpic}[width=0.47\textwidth]{figures/control-sample/ControlSampleRatio.pdf} 
%  \put(0,10){\rotatebox{-0}{\normalsize Expectation \& Prediction using single $\mu$ control sample (CS)}}
 \end{overpic}
 \begin{itemize}
  \item Ratio of $\mu$/ $e$ CS mostly 1. But espeically in low statistic regions differences expected
  \item More $\mu$ then $e$ CS in most relevant bins
 \end{itemize}

\end{frame}
\begin{frame}
  \frametitle{Comparison of both approaches: Simple 1 bin Example}
  \begin{itemize}
   \item Assume:
   \begin{itemize}
    \item Both methods predict same amount of lost-leptons ($\mu$ and $e$)
    \item Control-samples: 4 $\mu$, 1 $e$ (weight=1)
    \item Same weight for each $\mu$ \& $e$ 
   \end{itemize}

  \end{itemize}
  \scriptsize
  \begin{tabular}{l|r|r||r|r}

                                                  &           $\mu$            &           $e$  &             $\mu$            &           $e$  \\
\midrule 
     CS &                $4\pm2$ &             $1\pm1$ &              $4\pm2$ &             $1\pm1$  \\ \hline
      Pred.    &          $10\pm5$ (50\%) &              $10\pm10$(100\%)&              $20\pm10$(50\%)&                 $20\pm20$(100\%) \\
      Total Pred. &      $10\pm5$ (50\%) &              $10\pm10$(100\%)&              $20\pm10$(50\%)&                 $20\pm20$(100\%) \\
\bottomrule
\end{tabular}
\begin{itemize}
 \item Total lost-lepton prediction:
 \item Treating $\mu$ \& $e$ separately
 \begin{itemize}
  \item Total prediction: $20\pm11.2$
 \end{itemize}
 \item Using $\mu$ \& $e$ for both lost $\mu$ \& $e$
 \begin{itemize}
  \item Total prediction: $20\pm$
 \end{itemize}


\end{itemize}




\end{frame}






\subsection{SUSY}

\begin{frame}
\frametitle{Supersymmetry}
\normalsize
\begin{itemize}
 \item $\HT >500 \gev$
 \begin{itemize}
       \item Jets: $\pt>30\gev$, $|\eta|<2.5$
      \end{itemize}
 \item $\MHT >200 \gev$
  \begin{itemize}
       \item Jets: $\pt>30\gev$, $|\eta|<5.0$
      \end{itemize}
 \item $\NJets\ge 4$, \HT jets
 \item $\BTags$= {$0,1,2,\geq3$} CSVM ($>0.814$), $\pt>30\gev$
 \item $\deltaphi_{1,2,3}>0.5,0.5,0.3$
\item Veto Muons: \href{https://twiki.cern.ch/twiki/bin/view/CMSPublic/SWGuideMuonId\#Tight\_Muon}{2012 ``tight'' ID}: $p_T > 10$ GeV, $I_{rel}\; (\Delta R<0.4) < 0.2$    
    \item Veto Electrons: \href{https://twiki.cern.ch/twiki/bin/viewauth/CMS/CutBasedElectronIdentificationRun2\#CSA14\_selection\_conditions\_25ns}{Phys14 POG ID}:  $p_T > 10$ GeV, $I_{rel}\;
      (\Delta R<0.3) < 0.33 / 0.38$
      \item Under study:
      \begin{itemize}


    \item Taus: \href{https://indico.cern.ch/event/359233/contribution/4/material/slides/0.pdf}{Phys14 POG ID}: $p_T > 10$ GeV, $|\eta| < 2.3$,
      chargedIsoPtSum $(\Delta R<0.5)$ < 1.0 GeV (no neutral isolation yet)
    \item Isolated tracks: $p_T > 15$ GeV, $I_{rel}\;(\Delta R<0.3) < 0.1$ -- just charged candidates
      \end{itemize}
\end{itemize}
% \begin{block}{}
% \centering
% \Large
% \end{block}
\end{frame}


\section{Classical Lost-Lepton Method}


\begin{frame}
\frametitle{Lost-Lepton Background $\mu$ only}
 \begin{center}
 \begin{overpic}[width=1.0\textwidth]{figures/Sketches/LostLeptonSketch.pdf} \end{overpic}
 \end{center}
\end{frame}

\begin{frame}
\frametitle{Lost-Lepton Background $\mu$ only}
 \begin{center}
 \begin{overpic}[width=0.9\textwidth]{figures/Sketches/LostLeptonSketch_mu_pred.pdf} 
 \put(0,10){\rotatebox{-0}{\normalsize Expectation \& Prediction using single $\mu$ control-sample (CS)}}
 \end{overpic}

 \end{center}
\end{frame}


\begin{frame}
\frametitle{Lost-Lepton Background $\mu$ \& e}
 \begin{center}
 \begin{overpic}[width=0.9\textwidth]{figures/Sketches/LostLeptonSketch_ll.pdf} 
%  \put(0,10){\rotatebox{-0}{\normalsize Expectation \& Prediction using single $\mu$ control sample (CS)}}
 \end{overpic}

 \end{center}
\end{frame}


\begin{frame}
\frametitle{Lost-Lepton Background $\mu$ \& e}
 \begin{center}
 \begin{overpic}[width=0.80\textwidth]{figures/Sketches/LostLeptonSketch_mu_pred_full.pdf} 
  \put(0,-3){\rotatebox{-0}{\normalsize Expectation \& Prediction using single $\mu$ control-sample (CS)}}
 \end{overpic}

 \end{center}
\end{frame}

\begin{frame}
\frametitle{Lost-Lepton Background $\mu$ \& e}
 \begin{center}
 \begin{overpic}[width=0.80\textwidth]{figures/Sketches/LostLeptonSketch_e_pred_full.pdf} 
  \put(0,-3){\rotatebox{-0}{\normalsize Expectation \& Prediction using single e control-sample (CS)}}
 \end{overpic}

 \end{center}
\end{frame}
\begin{frame}
  \begin{center}
    \Large
     Backup
  \end{center}
\end{frame}


\begin{frame}
\frametitle{Baseline selection}
\normalsize
\begin{itemize}
 \item $\HT >500 \gev$
 \begin{itemize}
       \item Jets: $\pt>30\gev$, $|\eta|<2.5$ 
      \end{itemize}
 \item $\MHT >200 \gev$
  \begin{itemize}
       \item Jets: $\pt>30\gev$, $|\eta|<5.0$
      \end{itemize}
 \item $\NJets\ge 4$, \HT jets
 \item $\BTags$= {$0,1,2,\geq3$} CSVM ($>0.814$), $\pt>30\gev$
% \item $\deltaphi_{1,2,3}>0.5,0.5,0.3$
\item  \dphin $> 4.0$
\item Veto Muons: \href{https://twiki.cern.ch/twiki/bin/view/CMSPublic/SWGuideMuonId\#Tight\_Muon}{2012 ``tight'' ID}: $p_T > 10$ GeV, $I_{rel}\; (\Delta R<0.4) < 0.2$    
    \item Veto Electrons: \href{https://twiki.cern.ch/twiki/bin/viewauth/CMS/CutBasedElectronIdentificationRun2\#CSA14\_selection\_conditions\_25ns}{Phys14 POG ID}:  $p_T > 10$ GeV, $I_{rel}\;
      (\Delta R<0.3) < 0.33 / 0.38$
      \item Filters:
      \begin{itemize}
       \item Lose jet ID criteria for pf jets
      \end{itemize}

%       \item Under study:
%       \begin{itemize}
% 
% 
%     \item Taus: \href{https://indico.cern.ch/event/359233/contribution/4/material/slides/0.pdf}{Phys14 POG ID}: $p_T > 10$ GeV, $|\eta| < 2.3$,
%       chargedIsoPtSum $(\Delta R<0.5)$ < 1.0 GeV (no neutral isolation yet)
%     \item Isolated tracks: $p_T > 15$ GeV, $I_{rel}\;(\Delta R<0.3) < 0.1$ -- just charged candidates
%       \end{itemize}
\end{itemize}
% \begin{block}{}
% \centering
% \Large
% \end{block}
\end{frame}

% --------------------------------------------------

\setcounter{framenumber}{7}

\end{document}
