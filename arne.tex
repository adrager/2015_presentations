\documentclass{beamer}
%kdfj
\usetheme[secheader]{Boadilla}
\setbeamertemplate{footline} {
  %\leavevmode%
  \hbox{%
  \begin{beamercolorbox}[wd=.5\paperwidth,ht=2.25ex,dp=1ex,left]{author in head/foot}%
    \usebeamerfont{author in head/foot}\hspace*{2ex}\insertshortauthor~~(adraeger@cern.ch)
  \end{beamercolorbox}%
  \begin{beamercolorbox}[wd=.5\paperwidth,ht=2.25ex,dp=1ex,right]{date in head/foot}%
    \usebeamerfont{date in head/foot}\insertshorttitle,~
    \insertshortdate{}\hspace*{1em}
    \insertframenumber{} / \inserttotalframenumber\hspace*{2ex}
  \end{beamercolorbox}}%
  \vskip0pt%
}
\beamertemplatenavigationsymbolsempty

\usepackage[percent]{overpic}
\usepackage{tikz}
%\usetikzlibrary{positioning,fit,shapes.arrows,shapes.geometric,shapes.misc,shapes.multipart,calc,shadows}
\tikzstyle{every picture}+=[remember picture]
\usepackage{booktabs}
\usepackage{graphicx}
\usepackage{rotating}
\usepackage{wasysym}
\usepackage{marvosym}
\usepackage{amssymb}
\usepackage{xcolor}
\usepackage[normalem]{ulem}
\graphicspath{{../../logo/}{figures/}{../../graphic-common/}}

\input{definitions.tex}
\newcommand{\lib}[1]{\tiny #1}

% Title etc
\vskip2cm
\title[RA2/b Meeting]{Status of the \wpj \& \ttbar Working Group}
\subtitle{Classical Lost-Lepton Method\\ \MHT Extrapolation Approach\\ \hadtau Estimation Method\\ Lepton/Isolated Track Efficiencies}
\author[Arne-Rasmus~Dr\"ager]{
  Arne-Rasmus~Dr\"ager(Uni Hamburg)
}
\date[June 10, 2015]{June 10, 2015
  \vskip1cm
  \begin{center}
    \includegraphics[height=1.5cm]{Universitaet-Hamburg-Logo.jpg}
    \hskip8cm
    \includegraphics[height=1.5cm]{CMSlogo.jpeg}
  \end{center}
}

% pdflatex packages
\hypersetup{bookmarks=true}
\hypersetup{unicode=false}
\hypersetup{pdftitle={Lost-Lepton}}
\hypersetup{pdfauthor={Arne-Rasmus~Dr\"ager}}


\begin{document}
% ==================================================
% --------------------------------------------------
\begin{frame}
  \titlepage
\end{frame}

% \begin{frame}
%  \frametitle{Seed Sample Statistics}
%  \begin{itemize}
%   \item Estimate if we expect lack of \hadtau prediction in search bins
%  \end{itemize}
% 
%  
% \end{frame}


% \begin{frame}
%  \frametitle{To Do List:}
%  \begin{itemize}
%   \item Reverting simplification steps, put in again:
%   \begin{itemize}
%    \item Lepton veto
%    \item Use reconstructed \& isolated $\mu$ control sample
%    \item Apply out of acceptance correction
%   \end{itemize}
%   \item Merge effort by Rishi to scan entire template for each event to in crease precision  (bootstrap method)
%   \item Include \wpj
%   \item Include isolated track correction
%  \end{itemize}
%  
%   \begin{block}{}
%  \centering
%  We will try to sort out most of these topics this week
%  \end{block}
% \end{frame}
\section{Efficiencies}
\begin{frame}
 \begin{block}{}
 \centering
 \Large Lepton/Isolated Track Efficiencies
 \end{block}
\end{frame}

\begin{frame}
 \frametitle{Changes in respect to Wednesday talk}
 \begin{itemize}
  \item Tag\&Probe method technicallities chanes:
  \begin{itemize}
   \item Previously in a given event all combination of tag and probe are computed. Only the ones within $70\gev<M_{Z}<110\gev$ are considert. From these combinations only one was chosen.
   \item Chosing only one reduces the chance of picking actual good event. Especially when having many combinations as for going from chargedPFCands to reco leptons
   \item Now: Chose one pair each event which is closest to the $Z_{M}$
   \item Helps with signal over background espcially in problematic failing region ID/Reco improved!
  \end{itemize}

 \end{itemize}

\end{frame}


\subsection{Iso Efficiencies}

\begin{frame}
 \frametitle{Comparison \ttbar \& \wpj vs DY Tag \& Probe $\mu$ Iso Efficiencies}
   \begin{columns}

   \begin{column}{0.33\textwidth}
     \begin{itemize}
   \item $\mu$ Iso \ttbar \& \wpj eff. (truth info.)
  \end{itemize}
    \begin{tikzpicture}
    \node[anchor=south west,inner sep=0] (image) at (0,0) {\includegraphics[width=1.\textwidth]{figures/efficiencies/ttbar_wpj/MuIsoPTActivity.pdf}};
    \begin{scope}[x={(image.south east)},y={(image.north west)}]
%         \draw[red,ultra thick,rounded corners] (0.62,0.65) rectangle (0.78,0.75);
%         \draw[red,ultra thick,rounded corners] (0.60,0.01) rectangle (0.75,0.99); % coordinates unten links(x,y) oben rechts(x,y)
    \end{scope}
   \end{tikzpicture}
   \end{column}
   \begin{column}{0.33\textwidth}
   \begin{itemize}
    \item $\mu$ Iso DY eff. \\(Tag \& Probe)
   \end{itemize}

    \begin{tikzpicture}
    \node[anchor=south west,inner sep=0] (image) at (0,0) {\includegraphics[width=1.\textwidth]{figures/efficiencies/tagandprobe/MuIsoTagAndProbeMC.pdf}};
    \begin{scope}[x={(image.south east)},y={(image.north west)}]
%         \draw[red,ultra thick,rounded corners] (0.62,0.65) rectangle (0.78,0.75);
%         \draw[red,ultra thick,rounded corners] (0.60,0.01) rectangle (0.75,0.99); % coordinates unten links(x,y) oben rechts(x,y)
    \end{scope}
   \end{tikzpicture}
   \end{column}
           \begin{column}{0.33\textwidth}
   \begin{itemize}
    \item $\mu$ iso radio
   \end{itemize}

    \begin{tikzpicture}
     \node[anchor=south west,inner sep=0] (image) at (0,0) {\includegraphics[width=1.\textwidth]{figures/efficiencies/tagandprobe/MuIsoPTActivity_ratio.pdf}};
    \begin{scope}[x={(image.south east)},y={(image.north west)}]
%         \draw[red,ultra thick,rounded corners] (0.62,0.65) rectangle (0.78,0.75);
%         \draw[red,ultra thick,rounded corners] (0.60,0.01) rectangle (0.75,0.99); % coordinates unten links(x,y) oben rechts(x,y)
    \end{scope}
   \end{tikzpicture}
   \end{column}
  \end{columns}
\begin{itemize}
\item No visible changes in respect to Wednesday presentation.
\end{itemize}
\end{frame}
\begin{frame}
 \frametitle{Comparison \ttbar \& \wpj vs DY Tag \& Probe e Iso Efficiencies}
   \begin{columns}

   \begin{column}{0.33\textwidth}
     \begin{itemize}
   \item e Iso \ttbar \& \wpj eff. (truth info.)
  \end{itemize}
    \begin{tikzpicture}
    \node[anchor=south west,inner sep=0] (image) at (0,0) {\includegraphics[width=1.\textwidth]{figures/efficiencies/ttbar_wpj/ElecIsoPTActivity.pdf}};
    \begin{scope}[x={(image.south east)},y={(image.north west)}]
%         \draw[red,ultra thick,rounded corners] (0.62,0.65) rectangle (0.78,0.75);
%         \draw[red,ultra thick,rounded corners] (0.60,0.01) rectangle (0.75,0.99); % coordinates unten links(x,y) oben rechts(x,y)
    \end{scope}
   \end{tikzpicture}
   \end{column}
   \begin{column}{0.33\textwidth}
   \begin{itemize}
    \item e Iso DY eff. \\(Tag \& Probe)
   \end{itemize}

    \begin{tikzpicture}
    \node[anchor=south west,inner sep=0] (image) at (0,0) {\includegraphics[width=1.\textwidth]{figures/efficiencies/tagandprobe/ElecIsoTagAndProbeMC.pdf}};
    \begin{scope}[x={(image.south east)},y={(image.north west)}]
%         \draw[red,ultra thick,rounded corners] (0.62,0.65) rectangle (0.78,0.75);
%         \draw[red,ultra thick,rounded corners] (0.60,0.01) rectangle (0.75,0.99); % coordinates unten links(x,y) oben rechts(x,y)
    \end{scope}
   \end{tikzpicture}
   \end{column}
           \begin{column}{0.33\textwidth}
   \begin{itemize}
    \item e iso radio
   \end{itemize}

    \begin{tikzpicture}
     \node[anchor=south west,inner sep=0] (image) at (0,0) {\includegraphics[width=1.\textwidth]{figures/efficiencies/tagandprobe/ElecIsoPTActivity_ratio.pdf}};
    \begin{scope}[x={(image.south east)},y={(image.north west)}]
%         \draw[red,ultra thick,rounded corners] (0.62,0.65) rectangle (0.78,0.75);
%         \draw[red,ultra thick,rounded corners] (0.60,0.01) rectangle (0.75,0.99); % coordinates unten links(x,y) oben rechts(x,y)
    \end{scope}
   \end{tikzpicture}
   \end{column}
  \end{columns}
\begin{itemize}
\item No visible changes in respect to Wednesday presentation.
\end{itemize}
\end{frame}


\begin{frame}
 \frametitle{Comparison \ttbar \& \wpj vs DY Tag \& Probe $\mu$ Reco/ID Efficiencies}
   \begin{columns}

   \begin{column}{0.33\textwidth}
     \begin{itemize}
   \item $\mu$ reco/ID \ttbar \& \wpj eff. (truth info.)
  \end{itemize}
    \begin{tikzpicture}
    \node[anchor=south west,inner sep=0] (image) at (0,0) {\includegraphics[width=1.\textwidth]{figures/efficiencies/ttbar_wpj/MuRecoPTActivity.pdf}};
    \begin{scope}[x={(image.south east)},y={(image.north west)}]
%         \draw[red,ultra thick,rounded corners] (0.62,0.65) rectangle (0.78,0.75);
%         \draw[red,ultra thick,rounded corners] (0.60,0.01) rectangle (0.75,0.99); % coordinates unten links(x,y) oben rechts(x,y)
    \end{scope}
   \end{tikzpicture}
   \end{column}
   \begin{column}{0.33\textwidth}
   \begin{itemize}
    \item $\mu$ reco/ID DY eff. \\(Tag \& Probe)
   \end{itemize}

    \begin{tikzpicture}
    \node[anchor=south west,inner sep=0] (image) at (0,0) {\includegraphics[width=1.\textwidth]{figures/efficiencies/tagandprobe/MuRecoTagAndProbeMC.pdf}};
    \begin{scope}[x={(image.south east)},y={(image.north west)}]
%         \draw[red,ultra thick,rounded corners] (0.62,0.65) rectangle (0.78,0.75);
%         \draw[red,ultra thick,rounded corners] (0.60,0.01) rectangle (0.75,0.99); % coordinates unten links(x,y) oben rechts(x,y)
    \end{scope}
   \end{tikzpicture}
   \end{column}
           \begin{column}{0.33\textwidth}
   \begin{itemize}
    \item $\mu$ reco/ID radio
   \end{itemize}

    \begin{tikzpicture}
     \node[anchor=south west,inner sep=0] (image) at (0,0) {\includegraphics[width=1.\textwidth]{figures/efficiencies/tagandprobe/MuRecoPTActivity_ratio.pdf}};
    \begin{scope}[x={(image.south east)},y={(image.north west)}]
%         \draw[red,ultra thick,rounded corners] (0.62,0.65) rectangle (0.78,0.75);
%         \draw[red,ultra thick,rounded corners] (0.60,0.01) rectangle (0.75,0.99); % coordinates unten links(x,y) oben rechts(x,y)
    \end{scope}
   \end{tikzpicture}
   \end{column}
  \end{columns}
\begin{itemize}
 \item Staring with probe chargedPFCands with updated selection does not help.
\end{itemize}
\end{frame}

\begin{frame}
 \frametitle{Comparison \ttbar \& \wpj vs DY Tag \& Probe $\mu$ Reco/ID Efficiencies}
   \begin{columns}

   \begin{column}{0.33\textwidth}
     \begin{itemize}
   \item $\mu$ reco/ID \ttbar \& \wpj eff. (truth info.)
  \end{itemize}
    \begin{tikzpicture}
    \node[anchor=south west,inner sep=0] (image) at (0,0) {\includegraphics[width=1.\textwidth]{figures/efficiencies/ttbar_wpj/MuRecoPTActivity.pdf}};
    \begin{scope}[x={(image.south east)},y={(image.north west)}]
%         \draw[red,ultra thick,rounded corners] (0.62,0.65) rectangle (0.78,0.75);
%         \draw[red,ultra thick,rounded corners] (0.60,0.01) rectangle (0.75,0.99); % coordinates unten links(x,y) oben rechts(x,y)
    \end{scope}
   \end{tikzpicture}
   \end{column}
   \begin{column}{0.33\textwidth}
   \begin{itemize}
    \item $\mu$ reco/ID DY eff. \\(Tag \& Probe)
   \end{itemize}

    \begin{tikzpicture}
    \node[anchor=south west,inner sep=0] (image) at (0,0) {\includegraphics[width=1.\textwidth]{figures/efficiencies/tagandprobe/muReco_pfCandWithPDGID/MuRecoTagAndProbeMC.pdf}};
    \begin{scope}[x={(image.south east)},y={(image.north west)}]
%         \draw[red,ultra thick,rounded corners] (0.62,0.65) rectangle (0.78,0.75);
%         \draw[red,ultra thick,rounded corners] (0.60,0.01) rectangle (0.75,0.99); % coordinates unten links(x,y) oben rechts(x,y)
    \end{scope}
   \end{tikzpicture}
   \end{column}
           \begin{column}{0.33\textwidth}
   \begin{itemize}
    \item $\mu$ reco/ID radio
   \end{itemize}

    \begin{tikzpicture}
     \node[anchor=south west,inner sep=0] (image) at (0,0) {\includegraphics[width=1.\textwidth]{figures/efficiencies/tagandprobe/muReco_pfCandWithPDGID/MuRecoPTActivity_ratio.pdf}};
    \begin{scope}[x={(image.south east)},y={(image.north west)}]
%         \draw[red,ultra thick,rounded corners] (0.62,0.65) rectangle (0.78,0.75);
%         \draw[red,ultra thick,rounded corners] (0.60,0.01) rectangle (0.75,0.99); % coordinates unten links(x,y) oben rechts(x,y)
    \end{scope}
   \end{tikzpicture}
   \end{column}
  \end{columns}
\begin{itemize}
 \item Staring with probe chargedPFCands with pdgID=11 help a lot. But we are missing the fraction of not reconstructed $\mu$ which dont get to passing the pdgID on pfCands.
\end{itemize}
\end{frame}

\begin{frame}
 \frametitle{Comparison \ttbar \& \wpj vs DY Tag \& Probe e Reco/ID Efficiencies}
   \begin{columns}

   \begin{column}{0.33\textwidth}
     \begin{itemize}
   \item e reco/ID \ttbar \& \wpj eff. (truth info.)
  \end{itemize}
    \begin{tikzpicture}
    \node[anchor=south west,inner sep=0] (image) at (0,0) {\includegraphics[width=1.\textwidth]{figures/efficiencies/ttbar_wpj/ElecRecoPTActivity.pdf}};
    \begin{scope}[x={(image.south east)},y={(image.north west)}]
%         \draw[red,ultra thick,rounded corners] (0.62,0.65) rectangle (0.78,0.75);
%         \draw[red,ultra thick,rounded corners] (0.60,0.01) rectangle (0.75,0.99); % coordinates unten links(x,y) oben rechts(x,y)
    \end{scope}
   \end{tikzpicture}
   \end{column}
   \begin{column}{0.33\textwidth}
   \begin{itemize}
    \item e reco/ID DY eff. \\(Tag \& Probe)
   \end{itemize}

    \begin{tikzpicture}
    \node[anchor=south west,inner sep=0] (image) at (0,0) {\includegraphics[width=1.\textwidth]{figures/efficiencies/tagandprobe/ElecRecoTagAndProbeMC.pdf}};
    \begin{scope}[x={(image.south east)},y={(image.north west)}]
%         \draw[red,ultra thick,rounded corners] (0.62,0.65) rectangle (0.78,0.75);
%         \draw[red,ultra thick,rounded corners] (0.60,0.01) rectangle (0.75,0.99); % coordinates unten links(x,y) oben rechts(x,y)
    \end{scope}
   \end{tikzpicture}
   \end{column}
           \begin{column}{0.33\textwidth}
   \begin{itemize}
    \item e reco/ID radio
   \end{itemize}

    \begin{tikzpicture}
     \node[anchor=south west,inner sep=0] (image) at (0,0) {\includegraphics[width=1.\textwidth]{figures/efficiencies/tagandprobe/ElecRecoPTActivity_ratio.pdf}};
    \begin{scope}[x={(image.south east)},y={(image.north west)}]
%         \draw[red,ultra thick,rounded corners] (0.62,0.65) rectangle (0.78,0.75);
%         \draw[red,ultra thick,rounded corners] (0.60,0.01) rectangle (0.75,0.99); % coordinates unten links(x,y) oben rechts(x,y)
    \end{scope}
   \end{tikzpicture}
   \end{column}
  \end{columns}
\begin{itemize}
 \item Tag\&Probe eff. obtained using photons as probe object: $\gamma \rightarrow e_{reco/ID}$ 
 \item Low activity rather good agreement. At high activity we are missing the fraction of superclusters which are failing $E_{EM}/E{Hcal}$ (photon cuts)
\end{itemize}
\end{frame}

% \begin{frame}
%  \frametitle{ID/Reco Efficiency Confirmation Plans}
%  \begin{itemize}
%   \item $\mu/e$ ID efficiencies:
%   \begin{itemize}
%    \item Straight forward: Start with reconstructed lepton as probe
%   \end{itemize}
%   \item $\mu/e$ Reco efficiencies:
%   \begin{itemize}
%    \item Choice of probe object problematic: Want to have most 'basic' object e.g. for electrons: Start with photons (miniAOD photons start at $\pt>14 \gev\rightarrow$ use AOD)
%    \item Most basic available $\mu$ object: Charged track (miniAOD) large/overwhelming background probably hopeless
%    \item Large effort needed. If necessary fall back to POG efficiencies for scaling factors!
%   \end{itemize}
%  
% 
%  \end{itemize}
% \end{frame}

\begin{frame}
 \frametitle{Can we use POG reco/ID efficiencies instead?}
 \begin{itemize}
  \item POG will provide efficiencies not as a funcion of Activity and \pt
  \item Can we still use theirs for validation? Any intrinsic dependency when going to our extrem search regions?
  \item Check if reco/id eff. stable vs. \HT 
 \end{itemize}
\begin{tikzpicture}
     \node[anchor=south west,inner sep=0] (image) at (0,0) {\includegraphics[width=.49\textwidth]{figures/efficiencies/ttbar_wpj/MuRecoHT1D.pdf}};
    \begin{scope}[x={(image.south east)},y={(image.north west)}]
%         \draw[red,ultra thick,rounded corners] (0.62,0.65) rectangle (0.78,0.75);
%         \draw[red,ultra thick,rounded corners] (0.60,0.01) rectangle (0.75,0.99); % coordinates unten links(x,y) oben rechts(x,y)
    \end{scope}
   \end{tikzpicture}
\begin{tikzpicture}
     \node[anchor=south west,inner sep=0] (image) at (0,0) {\includegraphics[width=.49\textwidth]{figures/efficiencies/ttbar_wpj/ElecRecoHT1D.pdf}};
    \begin{scope}[x={(image.south east)},y={(image.north west)}]
%         \draw[red,ultra thick,rounded corners] (0.62,0.65) rectangle (0.78,0.75);
%         \draw[red,ultra thick,rounded corners] (0.60,0.01) rectangle (0.75,0.99); % coordinates unten links(x,y) oben rechts(x,y)
    \end{scope}
   \end{tikzpicture}
\begin{itemize}
 \item Electron and muon reconstruction and ID efficiencies are stable vs \HT
\end{itemize}

\end{frame}


\begin{frame}
 \frametitle{Can we use POG reco/ID efficiencies instead?}
\begin{tikzpicture}
     \node[anchor=south west,inner sep=0] (image) at (0,0) {\includegraphics[width=.49\textwidth]{figures/efficiencies/ttbar_wpj/MuRecoMHT1D.pdf}};
    \begin{scope}[x={(image.south east)},y={(image.north west)}]
%         \draw[red,ultra thick,rounded corners] (0.62,0.65) rectangle (0.78,0.75);
%         \draw[red,ultra thick,rounded corners] (0.60,0.01) rectangle (0.75,0.99); % coordinates unten links(x,y) oben rechts(x,y)
    \end{scope}
   \end{tikzpicture}
\begin{tikzpicture}
     \node[anchor=south west,inner sep=0] (image) at (0,0) {\includegraphics[width=.49\textwidth]{figures/efficiencies/ttbar_wpj/ElecRecoMHT1D.pdf}};
    \begin{scope}[x={(image.south east)},y={(image.north west)}]
%         \draw[red,ultra thick,rounded corners] (0.62,0.65) rectangle (0.78,0.75);
%         \draw[red,ultra thick,rounded corners] (0.60,0.01) rectangle (0.75,0.99); % coordinates unten links(x,y) oben rechts(x,y)
    \end{scope}
   \end{tikzpicture}
\begin{itemize}
 \item Electron and muon reconstruction and ID efficiencies are stable vs \MHT
\end{itemize}

\end{frame}



\begin{frame}
 \frametitle{Can we use POG reco/ID efficiencies instead?}
\begin{tikzpicture}
     \node[anchor=south west,inner sep=0] (image) at (0,0) {\includegraphics[width=.49\textwidth]{figures/efficiencies/ttbar_wpj/MuRecoNJets1D.pdf}};
    \begin{scope}[x={(image.south east)},y={(image.north west)}]
%         \draw[red,ultra thick,rounded corners] (0.62,0.65) rectangle (0.78,0.75);
%         \draw[red,ultra thick,rounded corners] (0.60,0.01) rectangle (0.75,0.99); % coordinates unten links(x,y) oben rechts(x,y)
    \end{scope}
   \end{tikzpicture}
\begin{tikzpicture}
     \node[anchor=south west,inner sep=0] (image) at (0,0) {\includegraphics[width=.49\textwidth]{figures/efficiencies/ttbar_wpj/ElecRecoNJets1D.pdf}};
    \begin{scope}[x={(image.south east)},y={(image.north west)}]
%         \draw[red,ultra thick,rounded corners] (0.62,0.65) rectangle (0.78,0.75);
%         \draw[red,ultra thick,rounded corners] (0.60,0.01) rectangle (0.75,0.99); % coordinates unten links(x,y) oben rechts(x,y)
    \end{scope}
   \end{tikzpicture}
\begin{itemize}
 \item Electron and muon reconstruction and ID efficiencies are stable vs \NJets
\end{itemize}

\end{frame}


\begin{frame}
 \frametitle{Can we use POG reco/ID efficiencies instead?}
\begin{tikzpicture}
     \node[anchor=south west,inner sep=0] (image) at (0,0) {\includegraphics[width=.49\textwidth]{figures/efficiencies/ttbar_wpj/MuRecoBTag1D.pdf}};
    \begin{scope}[x={(image.south east)},y={(image.north west)}]
%         \draw[red,ultra thick,rounded corners] (0.62,0.65) rectangle (0.78,0.75);
%         \draw[red,ultra thick,rounded corners] (0.60,0.01) rectangle (0.75,0.99); % coordinates unten links(x,y) oben rechts(x,y)
    \end{scope}
   \end{tikzpicture}
\begin{tikzpicture}
     \node[anchor=south west,inner sep=0] (image) at (0,0) {\includegraphics[width=.49\textwidth]{figures/efficiencies/ttbar_wpj/ElecRecoBTag1D.pdf}};
    \begin{scope}[x={(image.south east)},y={(image.north west)}]
%         \draw[red,ultra thick,rounded corners] (0.62,0.65) rectangle (0.78,0.75);
%         \draw[red,ultra thick,rounded corners] (0.60,0.01) rectangle (0.75,0.99); % coordinates unten links(x,y) oben rechts(x,y)
    \end{scope}
   \end{tikzpicture}
\begin{itemize}
 \item Electron and muon reconstruction and ID efficiencies are rather stable vs \BTags
\end{itemize}

\end{frame}


\subsection{Setup: Isolated Tracks(e/$\mu$)}
\begin{frame}
 \frametitle{Isolated Elec \& Muon Tracks}
 \begin{itemize}
 \item Muon, Electron Tracks:
 \begin{itemize}
  \item Charged PFCand, $\pt>5 GeV$, $\mt<100 GeV$ ask for pdgID=11,13
  \item Iso: $\Sigma ( \pt\text(Tracks)\Delta R<0.3 )/(\pt Track) < 0.2$ (with $dz<0.05$)
 \end{itemize}
 \item Tag \& Probe:
 \begin{itemize}
  \item Tag: Isolated $\mu$/e (high purity RA2b definition)
  \item Probe:
 \begin{itemize}
  \item Desirable Probe: chargedPFCands $\rightarrow$ iso Mu/Elec Track (not possible too high background)
  \item Instead Probe: chargedPFCands with pdgID=11,13 (cant test for pdgID)
  \item Still small statistics due to deriving efficiencies of isolated tracks to failing isolated leptons (not applied yet)
  \item Problem: No $\mt<100 GeV$ applicable (maybe treat tag lepton as neutrino emulate \wtolnu)
 \end{itemize}
  \end{itemize}
 \end{itemize}
\end{frame}

\begin{frame}
 \frametitle{Comparison \ttbar \& \wpj vs DY Tag \& Probe $\mu$ track Efficiencies}
   \begin{columns}

   \begin{column}{0.33\textwidth}
     \begin{itemize}
   \item $\mu$ track \ttbar \& \wpj eff. (truth info.)
  \end{itemize}
    \begin{tikzpicture}
    \node[anchor=south west,inner sep=0] (image) at (0,0) {\includegraphics[width=1.\textwidth]{figures/efficiencies/ttbar_wpj/IsoTrackMuPTActivity.pdf}};
    \begin{scope}[x={(image.south east)},y={(image.north west)}]
%         \draw[red,ultra thick,rounded corners] (0.62,0.65) rectangle (0.78,0.75);
%         \draw[red,ultra thick,rounded corners] (0.60,0.01) rectangle (0.75,0.99); % coordinates unten links(x,y) oben rechts(x,y)
    \end{scope}
   \end{tikzpicture}
   \end{column}
   \begin{column}{0.33\textwidth}
   \begin{itemize}
    \item $\mu$ track DY eff. \\(Tag \& Probe)
   \end{itemize}

    \begin{tikzpicture}
    \node[anchor=south west,inner sep=0] (image) at (0,0) {\includegraphics[width=1.\textwidth]{figures/efficiencies/tagandprobe/MuTrackTagAndProbeMC.pdf}};
    \begin{scope}[x={(image.south east)},y={(image.north west)}]
%         \draw[red,ultra thick,rounded corners] (0.62,0.65) rectangle (0.78,0.75);
%         \draw[red,ultra thick,rounded corners] (0.60,0.01) rectangle (0.75,0.99); % coordinates unten links(x,y) oben rechts(x,y)
    \end{scope}
   \end{tikzpicture}
   \end{column}
           \begin{column}{0.33\textwidth}
   \begin{itemize}
    \item $\mu$ track radio
   \end{itemize}

    \begin{tikzpicture}
     \node[anchor=south west,inner sep=0] (image) at (0,0) {\includegraphics[width=1.\textwidth]{figures/efficiencies/tagandprobe/MuIsoTrackPTActivity_ratio.pdf}};
    \begin{scope}[x={(image.south east)},y={(image.north west)}]
%         \draw[red,ultra thick,rounded corners] (0.62,0.65) rectangle (0.78,0.75);
%         \draw[red,ultra thick,rounded corners] (0.60,0.01) rectangle (0.75,0.99); % coordinates unten links(x,y) oben rechts(x,y)
    \end{scope}
   \end{tikzpicture}
   \end{column}
  \end{columns}
\begin{itemize}
 \item Tag: Isolated $\mu$
 \item Probe: chargedPFCands wiht pdgID=11 $\rightarrow$ isolated muon track
 \item The increase with \pt seems somehow wrong. I looked into this further and the fits are in improvalbe status. Work ongoing
\end{itemize}
\end{frame}


\begin{frame}
 \frametitle{Comparison \ttbar \& \wpj vs DY Tag \& Probe e track Efficiencies}
   \begin{columns}

   \begin{column}{0.33\textwidth}
     \begin{itemize}
   \item e track \ttbar \& \wpj eff. (truth info.)
  \end{itemize}
    \begin{tikzpicture}
    \node[anchor=south west,inner sep=0] (image) at (0,0) {\includegraphics[width=1.\textwidth]{figures/efficiencies/ttbar_wpj/IsoTrackElecPTActivity.pdf}};
    \begin{scope}[x={(image.south east)},y={(image.north west)}]
%         \draw[red,ultra thick,rounded corners] (0.62,0.65) rectangle (0.78,0.75);
%         \draw[red,ultra thick,rounded corners] (0.60,0.01) rectangle (0.75,0.99); % coordinates unten links(x,y) oben rechts(x,y)
    \end{scope}
   \end{tikzpicture}
   \end{column}
   \begin{column}{0.33\textwidth}
   \begin{itemize}
    \item e track DY eff. \\(Tag \& Probe)
   \end{itemize}

    \begin{tikzpicture}
    \node[anchor=south west,inner sep=0] (image) at (0,0) {\includegraphics[width=1.\textwidth]{figures/efficiencies/tagandprobe/ElecTrackTagAndProbeMC.pdf}};
    \begin{scope}[x={(image.south east)},y={(image.north west)}]
%         \draw[red,ultra thick,rounded corners] (0.62,0.65) rectangle (0.78,0.75);
%         \draw[red,ultra thick,rounded corners] (0.60,0.01) rectangle (0.75,0.99); % coordinates unten links(x,y) oben rechts(x,y)
    \end{scope}
   \end{tikzpicture}
   \end{column}
           \begin{column}{0.33\textwidth}
   \begin{itemize}
    \item e track radio
   \end{itemize}

    \begin{tikzpicture}
     \node[anchor=south west,inner sep=0] (image) at (0,0) {\includegraphics[width=1.\textwidth]{figures/efficiencies/tagandprobe/ElecIsoTrackPTActivity_ratio.pdf}};
    \begin{scope}[x={(image.south east)},y={(image.north west)}]
%         \draw[red,ultra thick,rounded corners] (0.62,0.65) rectangle (0.78,0.75);
%         \draw[red,ultra thick,rounded corners] (0.60,0.01) rectangle (0.75,0.99); % coordinates unten links(x,y) oben rechts(x,y)
    \end{scope}
   \end{tikzpicture}
   \end{column}
  \end{columns}
\begin{itemize}
 \item Tag: Isolated e
 \item Probe: chargedPFCands with pdgID=13 $\rightarrow$ isolated e track
 \item The increase with \pt seems somehow wrong. I looked into this further and the fits are in improvalbe status. Work ongoing (same as for iso mu track)
\end{itemize}
\end{frame}






\begin{frame}
 \begin{block}{}
 \centering
 \Large Backup
 \end{block}
\end{frame}



\subsection{Isolated $\pi$ Tracks}
\begin{frame}
\begin{columns}
 \begin{column}{0.65\textwidth}
  \begin{itemize}
   \item Tag\&Probe on chargedPFCands has too high bkg
   \item Idea: Use similarities of isolated $\mu/e$ \& $\pi$ tracks (to be evaluated)
  \end{itemize}

 \end{column}
 \begin{column}{0.35\textwidth}
    \begin{tikzpicture}
     \node[anchor=south west,inner sep=0] (image) at (0,0) {\includegraphics[width=1.\textwidth]{figures/Sketches/TauDecays.png}};
    \begin{scope}[x={(image.south east)},y={(image.north west)}]
%         \draw[red,ultra thick,rounded corners] (0.62,0.65) rectangle (0.78,0.75);
%         \draw[red,ultra thick,rounded corners] (0.60,0.01) rectangle (0.75,0.99); % coordinates unten links(x,y) oben rechts(x,y)
    \end{scope}
   \end{tikzpicture}
 \end{column}
\end{columns}
\begin{itemize}
 \item $\tau\rightarrow\pi^{-} + \nu (17\%)$ These should behave like $\mu/e$ tracks!? If so, give us rough idea on track eff. uncertainty
   \item $\tau\rightarrow\pi^{-} + 1/2\pi0 + \nu (53\%)$ Still only one charged track. Similar to $\tau\rightarrow\pi^{-} + \nu$ ? If so same approach, inflated uncertainty.
   \item What fraction of 3 prong $\tau$ get selected by isolated track? Rather small (10\%), if so, assigning high uncertainty would be practical.
\end{itemize}
\end{frame}



\section{Tag \& Probe}
\subsection{Setup: Leptons}
\begin{frame}
%  \frametitle{Mini Isolation (UCSB approach \href{https://indico.cern.ch/event/368826/contribution/3/material/slides/0.pdf}{Adam talk}) vs Classical Isolation}
\frametitle{Electron Muon Lepton Tag \& Probe Setup }
\begin{itemize}
 \item Muon:
 \begin{itemize}
  \item Reco/ID: "Tight" ID, $\pt>10 GeV$, $|\eta|<2.4$
  \item Iso: Mini Isolation: Max Cone: 0.2 Min Cone: 0.05 $\delta \beta I(rel)<0.2$
 \end{itemize}
  \item Electron:
 \begin{itemize}
  \item Reco/ID: "Veto" ID, $\pt>10 GeV$, $|\eta|<2.5$
  \item Iso: Mini Isolation: Max Cone: 0.2 Min Cone: 0.05 $\delta \beta I(rel)<0.1$
 \end{itemize}
 \item Tag \& Probe:
 \begin{itemize}
  \item Tag: Isolated $\mu$/e (high purity RA2b definition)
 \item ID:  (problematic direct comparison to eff from \ttbar \& \wpj)
 \begin{itemize}
  \item Problem: ChargedPFCands(slimmedPhotons) $\rightarrow$ Reco\&ID muon(electron)
  \item Problem: Muons: chargedPFCands, bad ratio signal / background, electron: slimmedPhotons miniAOD stores only down to 14 GeV $\rightarrow$ use AOD (not done)
 \end{itemize}
  \item Iso: (directly comparable to eff from \ttbar \& \wpj)
 \begin{itemize}
  \item Tag: Iso muon(Electron)
  \item Probe: Reco/ID muon(electron) $\rightarrow$ Iso muon(electron)
 
 \end{itemize}

 \end{itemize}
\end{itemize}
\end{frame}

% --------------------------------------------------

\setcounter{framenumber}{12}

\end{document}

