\documentclass{beamer}
%kdfj
\usetheme[secheader]{Boadilla}
\setbeamertemplate{footline} {
  %\leavevmode%
  \hbox{%
  \begin{beamercolorbox}[wd=.5\paperwidth,ht=2.25ex,dp=1ex,left]{author in head/foot}%
    \usebeamerfont{author in head/foot}\hspace*{2ex}\insertshortauthor~~(adraeger@cern.ch)
  \end{beamercolorbox}%
  \begin{beamercolorbox}[wd=.5\paperwidth,ht=2.25ex,dp=1ex,right]{date in head/foot}%
    \usebeamerfont{date in head/foot}\insertshorttitle,~
    \insertshortdate{}\hspace*{1em}
    \insertframenumber{} / \inserttotalframenumber\hspace*{2ex}
  \end{beamercolorbox}}%
  \vskip0pt%
}
\beamertemplatenavigationsymbolsempty

\usepackage[percent]{overpic}
\usepackage{tikz}
%\usetikzlibrary{positioning,fit,shapes.arrows,shapes.geometric,shapes.misc,shapes.multipart,calc,shadows}
\tikzstyle{every picture}+=[remember picture]
\usepackage{booktabs}
\usepackage{graphicx}
\usepackage{rotating}
\usepackage{wasysym}
\usepackage{marvosym}
\graphicspath{{../../logo/}{figures/}{../../graphic-common/}}

\input{definitions.tex}
\newcommand{\lib}[1]{\tiny #1}

% Title etc
\vskip2cm
\title[RA2/b Meeting]{Sketch of (Classical) Lost-Lepton Method}
\subtitle{Combining Single Muon and Electron Control Sample}
\author[Arne-Rasmus~Dr\"ager]{
  Arne-Rasmus~Dr\"ager (Uni Hamburg)
}
\date[February 05, 2015]{February 05, 2015
  \vskip1cm
  \begin{center}
    \includegraphics[height=1.5cm]{Universitaet-Hamburg-Logo.jpg}
    \hskip8cm
    \includegraphics[height=1.5cm]{CMSlogo.jpeg}
  \end{center}
}

% pdflatex packages
\hypersetup{bookmarks=true}
\hypersetup{unicode=false}
\hypersetup{pdftitle={Islated Tracks}}
\hypersetup{pdfauthor={Arne-Rasmus~Dr\"ager}}


\begin{document}
% ==================================================
% --------------------------------------------------
\begin{frame}
  \titlepage
\end{frame}



\section{Classical Lost-Lepton Method}
\begin{frame}
\frametitle{Outline}
\normalsize
\begin{itemize}
 \item Reminder: The lost-lepton background is estimated selecting a single $\mu$/e control-sample and weighting each event according to the lepton efficiencies
 \item The following slides illustrate the lost-lepton background composition and the full picture of the estimation method
 \begin{itemize}
 \item Slides 4,5: Simple case for only lost $\mu$ using a single $\mu$ control-sample
 \item Slides 6,7: Adding the lost electrons, still using only the single $\mu$ control-sample
 \item Slides 8: Same as slide 7 but using now the single electron control-sample
  \end{itemize}
  \item The final prediction of the lost-lepton method is a combination of the statistical uncorrelated predictions using the single electron and single muon control-sample
\end{itemize}
\end{frame}

\begin{frame}
\frametitle{Lost-Lepton Background $\mu$ only}
 \begin{center}
 \begin{overpic}[width=1.0\textwidth]{figures/Sketches/LostLeptonSketch.pdf} \end{overpic}
 \end{center}
\end{frame}

\begin{frame}
\frametitle{Lost-Lepton Background $\mu$ only}
 \begin{center}
 \begin{overpic}[width=0.9\textwidth]{figures/Sketches/LostLeptonSketch_mu_pred.pdf} 
 \put(0,10){\rotatebox{-0}{\normalsize Expectation \& Prediction using single $\mu$ control-sample (CS)}}
 \end{overpic}

 \end{center}
\end{frame}


\begin{frame}
\frametitle{Lost-Lepton Background $\mu$ \& e}
 \begin{center}
 \begin{overpic}[width=0.9\textwidth]{figures/Sketches/LostLeptonSketch_ll.pdf} 
%  \put(0,10){\rotatebox{-0}{\normalsize Expectation \& Prediction using single $\mu$ control sample (CS)}}
 \end{overpic}

 \end{center}
\end{frame}


\begin{frame}
\frametitle{Lost-Lepton Background $\mu$ \& e}
 \begin{center}
 \begin{overpic}[width=0.80\textwidth]{figures/Sketches/LostLeptonSketch_mu_pred_full.pdf} 
  \put(0,-3){\rotatebox{-0}{\normalsize Expectation \& Prediction using single $\mu$ control-sample (CS)}}
 \end{overpic}

 \end{center}
\end{frame}

\begin{frame}
\frametitle{Lost-Lepton Background $\mu$ \& e}
 \begin{center}
 \begin{overpic}[width=0.80\textwidth]{figures/Sketches/LostLeptonSketch_e_pred_full.pdf} 
  \put(0,-3){\rotatebox{-0}{\normalsize Expectation \& Prediction using single e control-sample (CS)}}
 \end{overpic}

 \end{center}
\end{frame}
\begin{frame}
  \begin{center}
    \Large
     Backup
  \end{center}
\end{frame}


\begin{frame}
\frametitle{Baseline selection}
\normalsize
\begin{itemize}
 \item $\HT >500 \gev$
 \begin{itemize}
       \item Jets: $\pt>30\gev$, $|\eta|<2.5$ 
      \end{itemize}
 \item $\MHT >200 \gev$
  \begin{itemize}
       \item Jets: $\pt>30\gev$, $|\eta|<5.0$
      \end{itemize}
 \item $\NJets\ge 4$, \HT jets
 \item $\BTags$= {$0,1,2,\geq3$} CSVM ($>0.814$), $\pt>30\gev$
% \item $\deltaphi_{1,2,3}>0.5,0.5,0.3$
\item  \dphin $> 4.0$
\item Veto Muons: \href{https://twiki.cern.ch/twiki/bin/view/CMSPublic/SWGuideMuonId\#Tight\_Muon}{2012 ``tight'' ID}: $p_T > 10$ GeV, $I_{rel}\; (\Delta R<0.4) < 0.2$    
    \item Veto Electrons: \href{https://twiki.cern.ch/twiki/bin/viewauth/CMS/CutBasedElectronIdentificationRun2\#CSA14\_selection\_conditions\_25ns}{Phys14 POG ID}:  $p_T > 10$ GeV, $I_{rel}\;
      (\Delta R<0.3) < 0.33 / 0.38$
      \item Filters:
      \begin{itemize}
       \item Lose jet ID criteria for pf jets
      \end{itemize}

%       \item Under study:
%       \begin{itemize}
% 
% 
%     \item Taus: \href{https://indico.cern.ch/event/359233/contribution/4/material/slides/0.pdf}{Phys14 POG ID}: $p_T > 10$ GeV, $|\eta| < 2.3$,
%       chargedIsoPtSum $(\Delta R<0.5)$ < 1.0 GeV (no neutral isolation yet)
%     \item Isolated tracks: $p_T > 15$ GeV, $I_{rel}\;(\Delta R<0.3) < 0.1$ -- just charged candidates
%       \end{itemize}
\end{itemize}
% \begin{block}{}
% \centering
% \Large
% \end{block}
\end{frame}

% --------------------------------------------------

\setcounter{framenumber}{7}

\end{document}
