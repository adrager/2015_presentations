\documentclass{beamer}
%kdfj
\usetheme[secheader]{Boadilla}
\setbeamertemplate{footline} {
  %\leavevmode%
  \hbox{%
  \begin{beamercolorbox}[wd=.5\paperwidth,ht=2.25ex,dp=1ex,left]{author in head/foot}%
    \usebeamerfont{author in head/foot}\hspace*{2ex}\insertshortauthor~~(adraeger@cern.ch)
  \end{beamercolorbox}%
  \begin{beamercolorbox}[wd=.5\paperwidth,ht=2.25ex,dp=1ex,right]{date in head/foot}%
    \usebeamerfont{date in head/foot}\insertshorttitle,~
    \insertshortdate{}\hspace*{1em}
    \insertframenumber{} / \inserttotalframenumber\hspace*{2ex}
  \end{beamercolorbox}}%
  \vskip0pt%
}
\beamertemplatenavigationsymbolsempty

\usepackage[percent]{overpic}
\usepackage{tikz}
%\usetikzlibrary{positioning,fit,shapes.arrows,shapes.geometric,shapes.misc,shapes.multipart,calc,shadows}
\tikzstyle{every picture}+=[remember picture]
\usepackage{booktabs}
\usepackage{graphicx}
\usepackage{rotating}
\usepackage{wasysym}
\usepackage{marvosym}
\graphicspath{{../../logo/}{figures/}{../../graphic-common/}}

\input{definitions.tex}
\newcommand{\lib}[1]{\tiny #1}

% Title etc
% \vskip2cm
\title[RA2/b Meeting]{Absch\"atzung des $W$ und \ttbar Untergrundes f\"ur die Suche nach neuer Physik im Endzustand mit Jets und fehlender Transversalenergie bei CMS}
 \subtitle{Studien mit 13TeV MC}
\author[Arne-Rasmus~Dr\"ager]{
  \underline{Arne-Rasmus~Dr\"ager}, Christian Sander \\(Uni Hamburg)
}
\date[March 05, 2015]{March 05, 2015
%   \vskip1cm
  \begin{center}
    \includegraphics[height=1.5cm]{Universitaet-Hamburg-Logo.jpg}
    \hskip8cm
    \includegraphics[height=1.5cm]{CMSlogo.jpeg}
  \end{center}
}

% pdflatex packages
\hypersetup{bookmarks=true}
\hypersetup{unicode=false}
\hypersetup{pdftitle={Lost-Lepton}}
\hypersetup{pdfauthor={Arne-Rasmus~Dr\"ager}}


\begin{document}
% ==================================================
% --------------------------------------------------
\begin{frame}
  \titlepage
\end{frame}
\section{Outline}
\begin{frame}
\frametitle{Outline}
\Large
\begin{itemize}
 \item Introduction:
 \begin{itemize}
  \item Supersymmetry
  \item Full Hadronic Analysis
 \end{itemize}
 \item \wpj \& \ttbar Lost-Lepton Estimation Method
 \begin{itemize}
  \item Basic Concept
  \item Closure Tests
  \item Efficiency Parametrization
 \end{itemize}
 \item Conclusion
\end{itemize}
\end{frame}

\section{Introduction}
\subsection{SUSY}
\begin{frame}
 \frametitle{Supersymmetry: An elegant extension of the standard model!}
 \begin{columns}
 \begin{column}{0.5\textwidth}

  \begin{itemize}
  \item SUSY advantages:
  \begin{itemize}
   \item Unification of gauge couplings possible
   \item Can provide dark matter candidate
   \item Solving hierarchy problem by canceling large SM mass contributions to Higgs mass
  \end{itemize}
  \item New light (TeV) scale particles favored
  \begin{itemize}
   \item \textbf{Production @ LHC possible!}
  \end{itemize}
 \end{itemize}
 \end{column}
 \begin{column}{0.5\textwidth}
  \begin{overpic}[width=.90\textwidth]{figures/Sketches/CouplingsUnification.png} \end{overpic}\\
  \begin{overpic}[width=.48\textwidth]{figures/Sketches/DarkMatter.png} \end{overpic}
  \begin{overpic}[width=.48\textwidth]{figures/Sketches/HiggsMassCancelation.png} \end{overpic}
 \end{column}

 \end{columns}

\end{frame}

\subsection{Full Hadronic Analysis}

\begin{frame}
 \begin{overpic}[width=.32\textwidth]{figures/feynmanDiagrams/T1qqqq.pdf} \end{overpic}
 \begin{overpic}[width=.32\textwidth]{figures/feynmanDiagrams/T1tttt.pdf} \end{overpic}
 \begin{overpic}[width=.32\textwidth]{figures/feynmanDiagrams/T1bbbb_feyn.pdf} \end{overpic}\\
 \begin{itemize}
  \item Focus on gluino topologies inspired by natural SUSY with R-parity conservation ($R=(-1)^{3(B-L)+2S}$)
  \item Select events with final state of only jets according to ``Baseline-Selection``:
  \begin{block}{}
  \begin{itemize}
   \item $\HT >500 \gev$ (Jets: $\pt>30\gev$, $|\eta|<2.5$)
   \item $\MHT >200 \gev$ (Jets: $\pt>30\gev$, $|\eta|<5.0$)
   \item $\NJets\ge 4$, $\BTags$= {$0,1,2,\geq3$},  \HT jets
   \item \dphin $> 4.0$ (QCD multi-jet rejection)
   \item Veto isolated $e/\mu$ (reject \ttbar \& \wpj)
  \end{itemize}
  \end{block}
  \item Increase sensitivity by dividing ``Baseline-Selection`` in $~72$ exclusive search bins of \HT, \MHT, \NJets \& \BTags  
 \end{itemize}
\end{frame}

\begin{frame}
 \begin{block}{}
 \centering
 \Large
 This talk: Estimation strategy of lost $e/\mu$\\ from \ttbar \& \wpj events
 \end{block}
%  \begin{itemize}
%   \item A similar method has been use on 7 \& 8 TeV
%   \item Extension and adaptation to exclusive \BTags bins and higher $\sqrt{s}$
%  \end{itemize}

\end{frame}
\section{Classical Lost-Lepton Method}
\subsection{Basic Concept}
\begin{frame}
 \begin{center}
\frametitle{Mainly \ttbar \& \wpj events where prompt electrons and muons are lost}

 \begin{overpic}[width=0.7\textwidth]{figures/Sketches/TotalLostLeptonBackgroundSmaller.png}
%    \put(0,0){\rotatebox{-0}{\normalsize  Select a control-sample (CS) of exactly one well isolated $e/\mu$ within acceptance cuts}}
%    \put(0,-10){\rotatebox{-0}{\normalsize  Weight each CS event according to efficiencies for each identification step}}
 \end{overpic}

 \end{center}
 \begin{itemize}
  \item Select a control-sample (CS) of exactly one well isolated $e/\mu$ within acceptance cuts
  \item Weight each CS event according to efficiencies for each identification step
 \end{itemize}

\end{frame}

\begin{frame}
\begin{itemize}
 \item Signal and other SM processes can contribute to e/$\mu$ control sample
 \item Suppress contamination by requiring trans. mass $\mt < 100 \gev$ \\
\end{itemize}
\vspace{0.5cm}
\hspace{0.5cm}$m_{T} = \sqrt{2 \cdot p_{T}(\mu/e)\cdot \met (1 - \cos(\Delta \Phi))}$

  \begin{columns}
    \begin{column}{0.5\textwidth}

      \begin{itemize}
      \item Removes about 10\% of e/$\mu$ CS due to:
        \begin{itemize}
        \item di-leptonic \ttbar decays
        \item Mismeasured jets
        \item Highly virtual W
        \end{itemize}
      \begin{centering}
      \end{centering}
      \item Corrected for as a function of lepton $p_{T}$ \& activity around the lepton (motivation \& definition on the next slides)
      \end{itemize}
      \vspace{0.3cm}
    \end{column}
    \begin{column}{0.5\textwidth}
      \centering
       \begin{overpic}[width=0.95\textwidth]{figures/control-sample/mtw.png}
       \put(53.93,80){\color{black}\line(0,-1){67}}
       \end{overpic}
    \end{column}
  \end{columns}
\end{frame}


\begin{frame}
\centering
 \frametitle{Predicting lost $e/\mu$ using single $\mu$ CS}
 \begin{center}
 \begin{overpic}[width=0.80\textwidth]{figures/Sketches/LostLeptonSketch_mu_pred_full.pdf}
%   \put(0,-3){\rotatebox{-0}{\normalsize Expectation \& Prediction using single $\mu$ control-sample (CS)}}
 \end{overpic}

 \end{center}
\end{frame}


\subsection{Closure Tests}

\begin{frame}
\frametitle{Comparison prediction to expectation on simulated events}
  \begin{columns}
    \begin{column}{0.5\textwidth}
     \centering
      \begin{overpic}[width=0.57\textwidth]{figures/lost-lepton-closure/Closure_Combined__HT__MCEx_vs_MuPrMTWDiLep+ElecPrMTWDiLep__Baseline.pdf}
     \end{overpic}
           \begin{overpic}[width=0.57\textwidth]{figures/lost-lepton-closure/Closure_Combined__MHT__MCEx_vs_MuPrMTWDiLep+ElecPrMTWDiLep__Baseline.pdf}
     \end{overpic}
    \end{column}
    \begin{column}{0.5\textwidth}
      \centering
            \begin{overpic}[width=0.57\textwidth]{figures/lost-lepton-closure/Closure_Combined__NJets__MCEx_vs_MuPrMTWDiLep+ElecPrMTWDiLep__Baseline.pdf}
     %     \put(90,90){\rotatebox{-45}{\scriptsize \Large Arne}}
     \end{overpic}
     \begin{overpic}[width=0.57\textwidth]{figures/lost-lepton-closure/Closure_Combined__BTags__MCEx_vs_MuPrMTWDiLep+ElecPrMTWDiLep__Baseline.pdf}
%       \begin{overpic}[width=0.57\textwidth]{figures/lost-lepton/Closure__NVtx__MCPrMTWDiLep_vs_MCEx__csa_Baseline.pdf}
      \end{overpic}
    \end{column}
  \end{columns}
  \begin{itemize}
  \item $\HT > 500 \gev$ , $\MHT>200 \gev$, $\NJets\ge 3$ , \dphin $> 4.0$
   \item Good closure but each exclusive search bin (72) closure relevant!
  \end{itemize}
\end{frame}

\begin{frame}
 \frametitle{Comparison prediction to expectation on simulated events for all 72 search bins}
  \begin{center}
 \begin{overpic}[width=0.70\textwidth]{figures/lost-lepton-closure/closureSearchbins.pdf}
     %     \put(90,90){\rotatebox{-45}{\scriptsize \Large Arne}}
     \end{overpic}
 \end{center}
 \begin{itemize}
  \item Overall good agreement for most of the search bins
  \item Low \BTags region needs further improvement (first 16 bins)
  \item Investigate fluctuations in regions of low control-sample statistics
 \end{itemize}
\end{frame}

\subsection{Efficiency Parametrization}
\begin{frame}
 \frametitle{Low statistics search region}
 \begin{itemize}
  \item Most sensitive search regions contain few $e/\mu$ control-sample events
  \item Important: \underline{Choice of efficiency parametrization!}
  \item Investigate different approaches:
  \begin{itemize}
   \item 1. Event topology based parametrization,\\ eg. \pt lepton \deltaR lepton closest jet
   \item 2. Search bin based efficiencies: Binned in \HT, \MHT, \NJets \& \BTags
  \end{itemize}
 \end{itemize}
 \noindent\makebox[\linewidth]{\rule{\textwidth}{0.4pt}}
 \begin{columns}
  \begin{column}{0.465\textwidth}
  Option 1
   \begin{itemize}
    \item Can be obtained directly from data using a Tag \& Probe method, if well chosen
    \item Difficult to parametrize event topology independent: Obtain from \Zll apply to \ttbar \wpj events
   \end{itemize}

  \end{column}
  \begin{column}{0.535\textwidth}
     Option 2
   \begin{itemize}
    \item Can not be obtained from data, needs to be validated indirectly via a Tag \& Probe method
    \item Captures \ttbar \& \wpj event kinematics according to quality of simulation
    \item Lags of statistics (72 search bins)
   \end{itemize}
  \end{column}

 \end{columns}
\end{frame}
\begin{frame}
\frametitle{Closer look at option 1}
    \begin{itemize}
   \item Classical approach of lepton \pt, (jet \pt) \& $\Delta R$ shows strong variation of efficiency and is strongly correlated with lepton isolation definition
   \item Severe problem when applying method to low statistical regions!
  \end{itemize}
  \begin{columns}
   \begin{column}{0.45\textwidth}
   \centering
    \small  Typical $\mu$ isolation efficiency
    \begin{overpic}[width=.99\textwidth]{figures/8TeV/DataTAPmuIso.pdf}
    \end{overpic}
   \end{column}
  \begin{column}{0.55\textwidth}
   Example: Region with 3 gen 1 CS event\\
    \begin{overpic} [width=0.45\textwidth]{figures/Sketches/deltaRLowStat.pdf}
      \end{overpic}
      \begin{overpic} [width=0.45\textwidth]{figures/Sketches/deltaRLowStatReco.pdf}
      \end{overpic}
     
  \end{column}
  \end{columns}
  \begin{itemize}
   \item Strong variation of efficiencies ($20-99\%$) lead to large fluctuations in low statistics regions
   \item Need to find parametrization:
   \begin{itemize}
    \item Capture event based topology in order to make transfer\\ \Zll to \ttbar, \wpj event kinematics
    \item More steady to avoid large fluctuations
   \end{itemize}
  \end{itemize}
\end{frame}

\begin{frame}
 \frametitle{New ansatz: Activity around lepton}
 \begin{itemize}
  \item Combine lepton \pt with subset of activity around the lepton in a fixed large cone size
  \begin{itemize}
\item Not orthogonal but also not fully correlated to isolation definition $\rightarrow$ subset of energy in cone around lepton
 \item Capture event kinematics in a more broader frame 'averaging' over range of strong efficiency drop in $\Delta R$
  \end{itemize}
 \end{itemize}
 
   \begin{columns}
   \begin{column}{0.5\textwidth}
   \centering
    \small  lepton \pt, (jet \pt) \& $\Delta R$
    \begin{overpic}[width=.99\textwidth]{figures/8TeV/DataTAPmuIso.pdf}
    \end{overpic}
   \end{column}
  \begin{column}{0.5\textwidth}
   \centering
    \small  lepton \pt, Activity
    \begin{overpic}[width=.99\textwidth]{figures/efficiencies/MuIsoPTActivity.pdf}
    \end{overpic}
     
  \end{column}
  \end{columns}
\section{Conclusion}
\end{frame}
\begin{frame}
 \frametitle{Conclusion}
 \begin{itemize}
  \item The lost-lepton method has been used for 7 \& 8 TeV analysis in a similar way
  \item Extension using also the single e CS in respect to 8 TeV has been accomplished 
  \item Exclusive \BTags regions haven been added as 4th dimention of search region definition
  \item Extensive studies have been performed on the efficiency parametrization
  \item Good performance already visible for wide range of phase space
  \item Ongoing work on:
  \begin{itemize}
   \item Optimization of efficiency binning \& choice of variables
   \item Specialized lepton isolation definition for high \HT \& \NJets regions
  \end{itemize}
 \end{itemize}

\end{frame}



\section{Backup}
\begin{frame}
  \begin{center}
    {\Large Additional Material}
  \end{center}
\end{frame}



% --------------------------------------------------

\setcounter{framenumber}{14}

\end{document}
