\documentclass{beamer}
%kdfj
\usetheme[secheader]{Boadilla}
\setbeamertemplate{footline} {
  %\leavevmode%
  \hbox{%
  \begin{beamercolorbox}[wd=.5\paperwidth,ht=2.25ex,dp=1ex,left]{author in head/foot}%
    \usebeamerfont{author in head/foot}\hspace*{2ex}\insertshortauthor~~(adraeger@cern.ch)
  \end{beamercolorbox}%
  \begin{beamercolorbox}[wd=.5\paperwidth,ht=2.25ex,dp=1ex,right]{date in head/foot}%
    \usebeamerfont{date in head/foot}\insertshorttitle,~
    \insertshortdate{}\hspace*{1em}
    \insertframenumber{} / \inserttotalframenumber\hspace*{2ex}
  \end{beamercolorbox}}%
  \vskip0pt%
}
\beamertemplatenavigationsymbolsempty

\usepackage[percent]{overpic}
\usepackage{tikz}
%\usetikzlibrary{positioning,fit,shapes.arrows,shapes.geometric,shapes.misc,shapes.multipart,calc,shadows}
\tikzstyle{every picture}+=[remember picture]
\usepackage{booktabs}
\usepackage{graphicx}
\usepackage{rotating}
\usepackage{wasysym}
\usepackage{marvosym}
\usepackage{amssymb}
\usepackage{xcolor}
\graphicspath{{../../logo/}{figures/}{../../graphic-common/}}

\input{definitions.tex}
\newcommand{\lib}[1]{\tiny #1}

% Title etc
\vskip2cm
\title[RA2/b Meeting]{Classical Lost-Lepton Background}
\subtitle{Tag \& Probe Efficiencies (Update) \\ Isolated Tracks Implementation in Method}
\author[Arne-Rasmus~Dr\"ager]{
  Arne-Rasmus~Dr\"ager(Uni Hamburg)
}
\date[June 04, 2015]{June 04, 2015
  \vskip1cm
  \begin{center}
    \includegraphics[height=1.5cm]{Universitaet-Hamburg-Logo.jpg}
    \hskip8cm
    \includegraphics[height=1.5cm]{CMSlogo.jpeg}
  \end{center}
}

% pdflatex packages
\hypersetup{bookmarks=true}
\hypersetup{unicode=false}
\hypersetup{pdftitle={Lost-Lepton}}
\hypersetup{pdfauthor={Arne-Rasmus~Dr\"ager}}


\begin{document}
% ==================================================
% --------------------------------------------------
\begin{frame}
  \titlepage
\end{frame}
\section{Lost-Lepton Method}
\begin{frame}
 \begin{block}{}
 \centering
 \Large Tag \& Probe Efficiencies for Lost-Lepton Background
 \end{block}
\end{frame}
\begin{frame}
 \begin{center}
\begin{tikzpicture}
    \node[anchor=south west,inner sep=0] (image) at (0,0) {\includegraphics[width=0.75\textwidth]{figures/Sketches/LostLeptonSketch_mu_pred_full.pdf}};
    \begin{scope}[x={(image.south east)},y={(image.north west)}]
%         \draw[red,ultra thick,rounded corners] (0.62,0.65) rectangle (0.78,0.75);
        \draw[red,ultra thick,rounded corners] (0.60,0.01) rectangle (0.75,0.99); % coordinates unten links(x,y) oben rechts(x,y)
            \draw[blue,ultra thick,rounded corners] (0.40,0.01) rectangle (0.55,0.99); % coordinates unten links(x,y) oben rechts(x,y)
    \end{scope}
\end{tikzpicture}

 \end{center}
 \begin{itemize}
  \item Deriving reco \& iso Efficiencies via Tag\&Probe
 \end{itemize}
\end{frame}

\section{Tag \& Probe}
\subsection{Setup: Leptons}
\begin{frame}
%  \frametitle{Mini Isolation (UCSB approach \href{https://indico.cern.ch/event/368826/contribution/3/material/slides/0.pdf}{Adam talk}) vs Classical Isolation}
\frametitle{Electron Muon Lepton Tag \& Probe Setup }
\begin{itemize}
 \item Muon:
 \begin{itemize}
  \item Reco/ID: "Tight" ID, $\pt>10 GeV$, $|\eta|<2.4$
  \item Iso: Mini Isolation: Max Cone: 0.2 Min Cone: 0.05 $\delta \beta I(rel)<0.2$
 \end{itemize}
  \item Electron:
 \begin{itemize}
  \item Reco/ID: "Veto" ID, $\pt>10 GeV$, $|\eta|<2.5$
  \item Iso: Mini Isolation: Max Cone: 0.2 Min Cone: 0.05 $\delta \beta I(rel)<0.1$
 \end{itemize}
 \item Tag \& Probe:
 \begin{itemize}
  \item Tag: Isolated $\mu$/e (high purity RA2b definition)
 \item ID:  (problematic direct comparison to eff from \ttbar \& \wpj)
 \begin{itemize}
  \item Problem: ChargedPFCands(slimmedPhotons) $\rightarrow$ Reco\&ID muon(electron)
  \item Problem: Muons: chargedPFCands, bad ratio signal / background, electron: slimmedPhotons miniAOD stores only down to 14 GeV $\rightarrow$ use AOD (not done)
 \end{itemize}
  \item Iso: (directly comparable to eff from \ttbar \& \wpj)
 \begin{itemize}
  \item Tag: Iso muon(Electron)
  \item Probe: Reco/ID muon(electron) $\rightarrow$ Iso muon(electron)
 
 \end{itemize}

 \end{itemize}
\end{itemize}
\end{frame}


% \begin{frame}
%  \frametitle{Tag \& Probe technicals}
%  \begin{itemize}
%   \item Test Several cuts to increase kinetically similarity to search region
%   \begin{itemize}
%    \item $\HT>500,400,200 \gev$
%    \item $\mindeltaphi >4.0, No Cut$ (Treat Tag leptons \pt as neutrino \pt $\rightarrow$ \MHT)
%    \item $\NJets\geq4$
%   \end{itemize}
%
%   \item Apply $\HT>500$ \& $\NJets\geq4$ (similar kinematics to baseline region)
%   \item Select (randomly) tag lepton from isolated lepton collection
%   \item Select all probe leptons except matched to tag
%   \item Compute for all invariant mass (Z mass)
%   \item Check property (ID/iso criteria)
%   \item Use electroweak code for fitting and eff. determination
%   \begin{itemize}
%    \item Fit function: gaussPlusCubic
%   \end{itemize}
%
%  \end{itemize}
% \end{frame}
\subsection{$\mu$ Efficiencies}
\begin{frame}
 \begin{block}{}
 \centering
 \Large Tag \& Probe $\mu$ Isolation/Reconstruction Efficiencies
 \end{block}
\end{frame}

\begin{frame}
 \frametitle{Comparison \ttbar \& \wpj vs DY Tag \& Probe $\mu$ Iso Efficiencies}
   \begin{columns}

   \begin{column}{0.33\textwidth}
     \begin{itemize}
   \item $\mu$ Iso \ttbar \& \wpj eff. (truth info.)
  \end{itemize}
    \begin{tikzpicture}
    \node[anchor=south west,inner sep=0] (image) at (0,0) {\includegraphics[width=1.\textwidth]{figures/Efficiencies/oldJEC/ttbarwpjTruth/MuIsoPTActivity.pdf}};
    \begin{scope}[x={(image.south east)},y={(image.north west)}]
%         \draw[red,ultra thick,rounded corners] (0.62,0.65) rectangle (0.78,0.75);
%         \draw[red,ultra thick,rounded corners] (0.60,0.01) rectangle (0.75,0.99); % coordinates unten links(x,y) oben rechts(x,y)
    \end{scope}
   \end{tikzpicture}
   \end{column}
   \begin{column}{0.33\textwidth}
   \begin{itemize}
    \item $\mu$ Iso DY eff. \\(Tag \& Probe)
   \end{itemize}

    \begin{tikzpicture}
    \node[anchor=south west,inner sep=0] (image) at (0,0) {\includegraphics[width=1.\textwidth]{figures/Efficiencies/oldJEC/tagAndProbe/MuIsoTagAndProbeMC.pdf}};
    \begin{scope}[x={(image.south east)},y={(image.north west)}]
%         \draw[red,ultra thick,rounded corners] (0.62,0.65) rectangle (0.78,0.75);
%         \draw[red,ultra thick,rounded corners] (0.60,0.01) rectangle (0.75,0.99); % coordinates unten links(x,y) oben rechts(x,y)
    \end{scope}
   \end{tikzpicture}
   \end{column}
           \begin{column}{0.33\textwidth}
   \begin{itemize}
    \item $\mu$ iso radio
   \end{itemize}

    \begin{tikzpicture}
     \node[anchor=south west,inner sep=0] (image) at (0,0) {\includegraphics[width=1.\textwidth]{figures/Efficiencies/oldJEC/tagAndProbe/MuIsoPTActivity_ratio.pdf}};
    \begin{scope}[x={(image.south east)},y={(image.north west)}]
%         \draw[red,ultra thick,rounded corners] (0.62,0.65) rectangle (0.78,0.75);
%         \draw[red,ultra thick,rounded corners] (0.60,0.01) rectangle (0.75,0.99); % coordinates unten links(x,y) oben rechts(x,y)
    \end{scope}
   \end{tikzpicture}
   \end{column}
  \end{columns}
\begin{itemize}
 \item Efficiencies obtained (using truth information) from \ttbar \& \wpj and DY are in good agreement
 \item Lepton \pt and Activity are sufficient topology independent to be transfered from DY to signal region! (Confirm Florent)
 \item Overall the efficiencies from DY are slightly higher. (No cuts applied to DY \ttbar \& \wpj baseline applied)
\end{itemize}
\end{frame}

\begin{frame}
 \frametitle{Comparison \ttbar \& \wpj vs DY-Truth $\mu$ Reco Efficiencies}
  \begin{columns}

   \begin{column}{0.33\textwidth}
     \begin{itemize}
   \item $\mu$ ID \ttbar \& \wpj eff. (truth info.)
  \end{itemize}
    \begin{tikzpicture}
    \node[anchor=south west,inner sep=0] (image) at (0,0) {\includegraphics[width=1.\textwidth]{figures/Efficiencies/oldJEC/ttbarwpjTruth/MuRecoPTActivity.pdf}};
    \begin{scope}[x={(image.south east)},y={(image.north west)}]
%         \draw[red,ultra thick,rounded corners] (0.62,0.65) rectangle (0.78,0.75);
%         \draw[red,ultra thick,rounded corners] (0.60,0.01) rectangle (0.75,0.99); % coordinates unten links(x,y) oben rechts(x,y)
    \end{scope}
   \end{tikzpicture}
   \end{column}
   \begin{column}{0.33\textwidth}
   \begin{itemize}
    \item $\mu$ ID DY eff.\\ (truth info.)
   \end{itemize}

    \begin{tikzpicture}
     \node[anchor=south west,inner sep=0] (image) at (0,0) {\includegraphics[width=1.\textwidth]{figures/Efficiencies/oldJEC/dytruth/MuRecoPTActivity_DY_noTagAndProbe.pdf}};
    \begin{scope}[x={(image.south east)},y={(image.north west)}]
%         \draw[red,ultra thick,rounded corners] (0.62,0.65) rectangle (0.78,0.75);
%         \draw[red,ultra thick,rounded corners] (0.60,0.01) rectangle (0.75,0.99); % coordinates unten links(x,y) oben rechts(x,y)
    \end{scope}
   \end{tikzpicture}
   \end{column}
        \begin{column}{0.33\textwidth}
   \begin{itemize}
    \item $\mu$ ID Tag \& Probe eff.
   \end{itemize}

    \begin{tikzpicture}
     \node[anchor=south west,inner sep=0] (image) at (0,0) {\includegraphics[width=1.\textwidth]{figures/Efficiencies/oldJEC/compareDYTruth_TTbarWPJTruth/MuRecoPTActivity_ratio.pdf}};
    \begin{scope}[x={(image.south east)},y={(image.north west)}]
%         \draw[red,ultra thick,rounded corners] (0.62,0.65) rectangle (0.78,0.75);
%         \draw[red,ultra thick,rounded corners] (0.60,0.01) rectangle (0.75,0.99); % coordinates unten links(x,y) oben rechts(x,y)
    \end{scope}
   \end{tikzpicture}
   \end{column}
  \end{columns}
\begin{itemize}
 \item Efficiencies obtained (using truth information) from \ttbar \& \wpj and DY are in good agreement.
 \item \ttbar \& \wpj eff. matching from all in acceptance (gen lepton) including non reco leptons
 \item \Zll eff. start with chargedPFCands/slimmedPhotons (slimmedMuon/slimmedElectron) (non reco excluded only test for ID)
\end{itemize}

\end{frame}


\begin{frame}
 \frametitle{Comparison \ttbar \& \wpj vs DY Tag \& Probe $\mu$ Reco Efficiencies}
   \begin{columns}

   \begin{column}{0.33\textwidth}
     \begin{itemize}
   \item $\mu$ Reco \ttbar \& \wpj eff. (truth info.)
  \end{itemize}
    \begin{tikzpicture}
    \node[anchor=south west,inner sep=0] (image) at (0,0) {\includegraphics[width=1.\textwidth]{figures/Efficiencies/oldJEC/ttbarwpjTruth/MuRecoPTActivity.pdf}};
    \begin{scope}[x={(image.south east)},y={(image.north west)}]
%         \draw[red,ultra thick,rounded corners] (0.62,0.65) rectangle (0.78,0.75);
%         \draw[red,ultra thick,rounded corners] (0.60,0.01) rectangle (0.75,0.99); % coordinates unten links(x,y) oben rechts(x,y)
    \end{scope}
   \end{tikzpicture}
   \end{column}
   \begin{column}{0.33\textwidth}
   \begin{itemize}
    \item $\mu$ Reco DY eff. \\(Tag \& Probe)
   \end{itemize}

    \begin{tikzpicture}
    \node[anchor=south west,inner sep=0] (image) at (0,0) {\includegraphics[width=1.\textwidth]{figures/Efficiencies/oldJEC/tagAndProbe/MuRecoTagAndProbeMC.pdf}};
    \begin{scope}[x={(image.south east)},y={(image.north west)}]
%         \draw[red,ultra thick,rounded corners] (0.62,0.65) rectangle (0.78,0.75);
%         \draw[red,ultra thick,rounded corners] (0.60,0.01) rectangle (0.75,0.99); % coordinates unten links(x,y) oben rechts(x,y)
    \end{scope}
   \end{tikzpicture}
   \end{column}
           \begin{column}{0.33\textwidth}
   \begin{itemize}
    \item $\mu$ Reco radio
   \end{itemize}

    \begin{tikzpicture}
     \node[anchor=south west,inner sep=0] (image) at (0,0) {\includegraphics[width=1.\textwidth]{figures/Efficiencies/oldJEC/tagAndProbe/MuRecoPTActivity_ratio.pdf}};
    \begin{scope}[x={(image.south east)},y={(image.north west)}]
%         \draw[red,ultra thick,rounded corners] (0.62,0.65) rectangle (0.78,0.75);
%         \draw[red,ultra thick,rounded corners] (0.60,0.01) rectangle (0.75,0.99); % coordinates unten links(x,y) oben rechts(x,y)
    \end{scope}
   \end{tikzpicture}
   \end{column}
  \end{columns}
\begin{itemize}
 \item Starting from probe charged pfCand too high background to signal event ratio in failing. (Here only DY sample used full SM process a lot higher background)
 \item Need to apply more cuts on probe to achieve higher purity. (follow official approach)
\end{itemize}
\end{frame}
\subsection{e Efficiencies}
\begin{frame}
 \begin{block}{}
 \centering
 \Large Tag \& Probe e Efficiencies
 \end{block}
\end{frame}


\begin{frame}
 \frametitle{Comparison \ttbar \& \wpj vs DY Tag \& Probe e Iso Efficiencies}
   \begin{columns}

   \begin{column}{0.33\textwidth}
     \begin{itemize}
   \item e Iso \ttbar \& \wpj eff. (truth info.)
  \end{itemize}
    \begin{tikzpicture}
    \node[anchor=south west,inner sep=0] (image) at (0,0) {\includegraphics[width=1.\textwidth]{figures/Efficiencies/oldJEC/ttbarwpjTruth/ElecIsoPTActivity.pdf}};
    \begin{scope}[x={(image.south east)},y={(image.north west)}]
%         \draw[red,ultra thick,rounded corners] (0.62,0.65) rectangle (0.78,0.75);
%         \draw[red,ultra thick,rounded corners] (0.60,0.01) rectangle (0.75,0.99); % coordinates unten links(x,y) oben rechts(x,y)
    \end{scope}
   \end{tikzpicture}
   \end{column}
   \begin{column}{0.33\textwidth}
   \begin{itemize}
    \item e Iso DY eff. \\(Tag \& Probe)
   \end{itemize}

    \begin{tikzpicture}
    \node[anchor=south west,inner sep=0] (image) at (0,0) {\includegraphics[width=1.\textwidth]{figures/Efficiencies/oldJEC/tagAndProbe/ElecIsoTagAndProbeMC.pdf}};
    \begin{scope}[x={(image.south east)},y={(image.north west)}]
%         \draw[red,ultra thick,rounded corners] (0.62,0.65) rectangle (0.78,0.75);
%         \draw[red,ultra thick,rounded corners] (0.60,0.01) rectangle (0.75,0.99); % coordinates unten links(x,y) oben rechts(x,y)
    \end{scope}
   \end{tikzpicture}
   \end{column}
           \begin{column}{0.33\textwidth}
   \begin{itemize}
    \item e iso radio
   \end{itemize}

    \begin{tikzpicture}
     \node[anchor=south west,inner sep=0] (image) at (0,0) {\includegraphics[width=1.\textwidth]{figures/Efficiencies/oldJEC/tagAndProbe/ElecIsoPTActivity_ratio.pdf}};
    \begin{scope}[x={(image.south east)},y={(image.north west)}]
%         \draw[red,ultra thick,rounded corners] (0.62,0.65) rectangle (0.78,0.75);
%         \draw[red,ultra thick,rounded corners] (0.60,0.01) rectangle (0.75,0.99); % coordinates unten links(x,y) oben rechts(x,y)
    \end{scope}
   \end{tikzpicture}
   \end{column}
  \end{columns}
\begin{itemize}
 \item Efficiencies obtained (using truth information) from \ttbar \& \wpj and DY are in good agreement
 \item Lepton \pt and Activity are sufficient topology independent to be transfered from DY to signal region! (Confirm Florent)
 \item Overall the efficiencies from DY are slightly higher. (No cuts applied to DY \ttbar \& \wpj baseline applied)
\end{itemize}
\end{frame}

\begin{frame}
 \frametitle{Comparison \ttbar \& \wpj vs DY-Truth e Reco Efficiencies}
  \begin{columns}

   \begin{column}{0.33\textwidth}
     \begin{itemize}
   \item e ID \ttbar \& \wpj eff. (truth info.)
  \end{itemize}
    \begin{tikzpicture}
    \node[anchor=south west,inner sep=0] (image) at (0,0) {\includegraphics[width=1.\textwidth]{figures/Efficiencies/oldJEC/ttbarwpjTruth/ElecRecoPTActivity.pdf}};
    \begin{scope}[x={(image.south east)},y={(image.north west)}]
%         \draw[red,ultra thick,rounded corners] (0.62,0.65) rectangle (0.78,0.75);
%         \draw[red,ultra thick,rounded corners] (0.60,0.01) rectangle (0.75,0.99); % coordinates unten links(x,y) oben rechts(x,y)
    \end{scope}
   \end{tikzpicture}
   \end{column}
   \begin{column}{0.33\textwidth}
   \begin{itemize}
    \item e ID DY eff.\\ (truth info.)
   \end{itemize}

    \begin{tikzpicture}
     \node[anchor=south west,inner sep=0] (image) at (0,0) {\includegraphics[width=1.\textwidth]{figures/Efficiencies/oldJEC/dytruth/ElecRecoPTActivity_DY_noTagAndProbe.pdf}};
    \begin{scope}[x={(image.south east)},y={(image.north west)}]
%         \draw[red,ultra thick,rounded corners] (0.62,0.65) rectangle (0.78,0.75);
%         \draw[red,ultra thick,rounded corners] (0.60,0.01) rectangle (0.75,0.99); % coordinates unten links(x,y) oben rechts(x,y)
    \end{scope}
   \end{tikzpicture}
   \end{column}
        \begin{column}{0.33\textwidth}
   \begin{itemize}
    \item e ID Tag \& Probe eff.
   \end{itemize}

    \begin{tikzpicture}
     \node[anchor=south west,inner sep=0] (image) at (0,0) {\includegraphics[width=1.\textwidth]{figures/Efficiencies/oldJEC/compareDYTruth_TTbarWPJTruth/ElecRecoPTActivity_ratio.pdf}};
    \begin{scope}[x={(image.south east)},y={(image.north west)}]
%         \draw[red,ultra thick,rounded corners] (0.62,0.65) rectangle (0.78,0.75);
%         \draw[red,ultra thick,rounded corners] (0.60,0.01) rectangle (0.75,0.99); % coordinates unten links(x,y) oben rechts(x,y)
    \end{scope}
   \end{tikzpicture}
   \end{column}
  \end{columns}
\begin{itemize}
 \item Efficiencies obtained (using truth information) from \ttbar \& \wpj and DY are in good agreement.
 \item \ttbar \& \wpj eff. matching from all in acceptance (gen lepton) including non reco leptons
 \item \Zll eff. start with chargedPFCands/slimmedPhotons (slimmedElecon/slimmedElectron) (non reco excluded only test for ID)
\end{itemize}

\end{frame}


\begin{frame}
 \frametitle{Comparison \ttbar \& \wpj vs DY Tag \& Probe e Reco Efficiencies}
   \begin{columns}

   \begin{column}{0.33\textwidth}
     \begin{itemize}
   \item e Reco \ttbar \& \wpj eff. (truth info.)
  \end{itemize}
    \begin{tikzpicture}
    \node[anchor=south west,inner sep=0] (image) at (0,0) {\includegraphics[width=1.\textwidth]{figures/Efficiencies/oldJEC/ttbarwpjTruth/ElecRecoPTActivity.pdf}};
    \begin{scope}[x={(image.south east)},y={(image.north west)}]
%         \draw[red,ultra thick,rounded corners] (0.62,0.65) rectangle (0.78,0.75);
%         \draw[red,ultra thick,rounded corners] (0.60,0.01) rectangle (0.75,0.99); % coordinates unten links(x,y) oben rechts(x,y)
    \end{scope}
   \end{tikzpicture}
   \end{column}
   \begin{column}{0.33\textwidth}
   \begin{itemize}
    \item e Reco DY eff. \\(Tag \& Probe)
   \end{itemize}

    \begin{tikzpicture}
    \node[anchor=south west,inner sep=0] (image) at (0,0) {\includegraphics[width=1.\textwidth]{figures/Efficiencies/oldJEC/tagAndProbe/ElecRecoTagAndProbeMC.pdf}};
    \begin{scope}[x={(image.south east)},y={(image.north west)}]
%         \draw[red,ultra thick,rounded corners] (0.62,0.65) rectangle (0.78,0.75);
%         \draw[red,ultra thick,rounded corners] (0.60,0.01) rectangle (0.75,0.99); % coordinates unten links(x,y) oben rechts(x,y)
    \end{scope}
   \end{tikzpicture}
   \end{column}
           \begin{column}{0.33\textwidth}
   \begin{itemize}
    \item e Reco radio
   \end{itemize}

    \begin{tikzpicture}
     \node[anchor=south west,inner sep=0] (image) at (0,0) {\includegraphics[width=1.\textwidth]{figures/Efficiencies/oldJEC/tagAndProbe/ElecRecoPTActivity_ratio.pdf}};
    \begin{scope}[x={(image.south east)},y={(image.north west)}]
%         \draw[red,ultra thick,rounded corners] (0.62,0.65) rectangle (0.78,0.75);
%         \draw[red,ultra thick,rounded corners] (0.60,0.01) rectangle (0.75,0.99); % coordinates unten links(x,y) oben rechts(x,y)
    \end{scope}
   \end{tikzpicture}
   \end{column}
  \end{columns}
\begin{itemize}
 \item Starting from probe charged pfCand too high background to signal event ratio in failing. (Here only DY sample used full SM process a lot higher background)
 \item Need to apply more cuts on probe to achieve higher purity. (follow official approach)
\end{itemize}
\end{frame}

\subsection{IsoTrack Implementation}
\begin{frame}
 \begin{block}{}
 \centering
 \Large Isolated e/$\mu$ Tracks: Implementation in Lost-Lepton Method
 \end{block}
\end{frame}

\begin{frame}
 \frametitle{Classical Lost-Lepton Estimation with Isotrack Reduction}
 \begin{itemize}
  \item Isolated track (mainly pdgID=11,13) reduce lost-lepton background even further 22\% reduction (on baseline)
  \item Current idea of including this reduction in method:
  \begin{itemize}
   \item Apply full classical lost-lepton method. In the very end correct for isolated track reduction of lost-lepton background ($C_{IsoTrack}$)
   \item Problem: Mixture of (classical) lepton efficiencies and isolated track efficiencies. (note: correcting relative to single lep. control-sample)
   \item Correction of lost-lepton background depends not only on isolated track efficiencies but also on lepton eff.
   \item $p_{full}(\epsilon_{ll},\epsilon_{IsoTrackRel}) = p_{ll}(\epsilon_{ll}) * (1-C_{IsoTrack}(\epsilon_{ll},\epsilon_{IsoTrackRel}))$
   \item $C_{IsoTrackRel}(\epsilon_{ll},\epsilon_{isotrack}) = \frac{(1-\epsilon_{ll}) * \epsilon_{IsoTrack}}{\epsilon_{ll}} (1-\epsilon_{ll}) * \epsilon_{IsoTrack} / \epsilon_{ll}$
   \item Sample to derive $C_{IsoTrackRel}$ consists of only failing (lost-lepton) events.
   \item Deriving/estimating uncertainties complicated. Note: $\epsilon_{ll}(\epsilon_{e/\mu acc,reco,iso,\mt})$
  \end{itemize}

 \end{itemize}

\end{frame}
\subsection{Setup: Isolated Tracks(e/$\mu$)}
\begin{frame}
 \frametitle{Isolated Track: Elec \& Muon Tracks}
 \begin{itemize}
 \item Muon, Electron Tracks:
 \begin{itemize}
  \item Charged PFCand, $\pt>5 GeV$, $\mt<100 GeV$ ask for pdgID=11,13
  \item Iso: $\Sigma ( \pt\text(Tracks)\Delta R<0.3 )/(\pt Track) < 0.2$ (with $dz<0.05$)
 \end{itemize}
 \item Tag \& Probe:
 \begin{itemize}
  \item Tag: Isolated $\mu$/e (high purity RA2b definition)
  \item Probe:
 \begin{itemize}
  \item Probe: chargedPFCands $\rightarrow$ iso Mu/Elec Track
  \item Problem: Large amount chargedPFCands, bad ratio signal / background
  \item Problem: Small statistics due to deriving efficiencies of isolated tracks to failing isolated leptons (not applied yet)
  \item Problem: No $\mt<100 GeV$ applicable
 \end{itemize}
  \end{itemize}
 \end{itemize}

 
\end{frame}
\subsection{IsoTrack Tag\&Probe}
\begin{frame}
 \begin{block}{}
 \centering
 \Large Isolated e Tracks: First try with Tag\&Probe \\
 \small $\mu$ similar not shown here
 \end{block}
\end{frame}

\begin{frame}
 \frametitle{First Look at Isolated Electron Tracks}
 \begin{center}
 \begin{columns}
  \begin{column}{0.4\textwidth}
      \begin{tikzpicture}
   \node[anchor=south west,inner sep=0] (image) at (0,0) {\includegraphics[width=1.\textwidth]{figures/Efficiencies/newJEC/tagAndProbe/ElecTrackTagAndProbeMC.pdf}};
   \begin{scope}[x={(image.south east)},y={(image.north west)}]
   \end{scope}
 \end{tikzpicture}
  \end{column}
  \begin{column}{0.3\textwidth}
      \begin{tikzpicture}
   \node[anchor=south west,inner sep=0] (image) at (0,0) {\includegraphics[width=1.\textwidth]{figures/Efficiencies/newJEC/tagAndProbe/ElecTrackBinExample.pdf}};
   \begin{scope}[x={(image.south east)},y={(image.north west)}]
   \end{scope}
 \end{tikzpicture}
   \end{column}
   \begin{column}{0.3\textwidth}
       \begin{tikzpicture}
   \node[anchor=south west,inner sep=0] (image) at (0,0) {\includegraphics[width=1.\textwidth]{figures/Efficiencies/newJEC/tagAndProbe/ElecTrackBinExampleHighPtHighActivity.pdf}};
   \begin{scope}[x={(image.south east)},y={(image.north west)}]
   \end{scope}
 \end{tikzpicture}
  \end{column}
 \end{columns}
 \begin{itemize}
 \item Starting with any charged track (applying only $\pt, \eta$ cuts) as probe
  \item Most bins too bad background/signal events (see middle plot)
  \item Adaptation of background fit function can help in some bins, BUT here only DY $\HT>400GeV$ sample used. Expected a lot worse contamination in data.
  \item This lose definition of probe candidates has too high background/signal ratio.
  \item Apply some sort of preselection (can test for isolation by starting with pdgID lepton track but excluding pdgID determination efficiency!)
 \end{itemize}



 \end{center}


\end{frame}


\begin{frame}
 \frametitle{Conclusion}
 \begin{itemize}
  \item Bug fixed (moved to eGamma maintained tools)
  \item Lepton Isolation eff:
  \begin{itemize}
   \item Still residual difference visible. Try applying $\HT>500GeV$ cut more busy environment
  \end{itemize}
  \item Lepton ID/Reco eff:
  \begin{itemize}
   \item Starting with slimmedMuon/slimmedElecon as probe tests only for ID criteria not sufficient
   \item Starting with chargedTrack too high background/signal ratio in failing collection (for muons)
   \item Starting with slimmedPhoton starts only at 14 GeV (in miniAOD) need to move to AOD
   \end{itemize}
   \item Lepton Tracks:
   \begin{itemize}
   \item Start with charged tracks (only $\pt, \eta$ cuts applied)
    \item Suffers from very bad background/signal ratio even when looking at 'pure' DY sample
    \item Need to start with some sort of preselection maybe pdgID already applied (cant test for pdgID efficiency)
   \end{itemize}


 
 \end{itemize}
 

\end{frame}

\begin{frame}
 \begin{block}{}
 \centering
 \Large Backup
 \end{block}
\end{frame}


% --------------------------------------------------

\setcounter{framenumber}{18}

\end{document}

